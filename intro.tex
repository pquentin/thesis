% vim: set spelllang=fr:
\chapter{Introduction}
\label{ch:intro}

\section{L'annotation en rôles sémantiques}

L'annotation en rôles sémantiques est une tâche d'analyse sémantique aux
applications nombreuses, telles que l'extraction et la recherche d'information,
la traduction automatique, ou encore le résumé automatique de textes.

Elle répond à la question « Qui a fait Quoi à Qui, Comment, Où et Quand ? ».
Prenons pour exemple \emph{Mrs. Aouda essaya vainement de retenir Mr. Fogg}
(extrait du \emph{Tour du monde en quatre-vingts jours} de Jules Verne).
L'annotation en rôles sémantique déterminera que la phrase correspond à une
situation de tentative (grâce à la présence du verbe « essayer »), puis
déterminera parmi les syntagmes liés aux verbes quel est l'Agent, l'Activité
tentée, et le Résultat (\emph{vainement}). Ainsi, le résultat de l'annotation
serait :

\begin{figure}[htbl]
    \centering
    \begin{tabular}{cccc}
      [Agent]  & \textbf{Tentative} & [Résultat]  & [Activité]           \tabularnewline
    Mrs. Aouda & \textbf{essaya}  & vainement  & de retenir Mr. Fogg. \tabularnewline
    \end{tabular}
    \caption{\label{fig:introsrl}Le verbe \emph{essayer} déclenche la situation \emph{Tentative}.
    Les différents syntagmes liés au verbe jouent chacun un rôle sémantique.}
\end{figure}

% TODO définir frame

Différents informations sont disponibles après l'annotation en rôles sémantiques.

\begin{itemize}
    \item Le prédicat ayant déclenché la frame est identifié. Dans la
        figure~\ref{fig:introsrl}, c'est un verbe, mais d'autres parties du
        discours peuvent déclencher une frame.
    \item La frame est identifiée, ici \emph{Tentative}.
    \item Enfin, les rôles exprimés sont annotés. Par exemple, \emph{Mrs.
        Aouda} est l'Agent.
\end{itemize}

\section{Applications}

Selon \cite{gildea2002automatic}, l'annotation en rôles sémantiques est une
évolution naturelle de certains travaux sur l'extraction d'information où les
systèmes traitent des situations très spécifiques, par exemple la détection de
résultats d'évènements sportifs ou la détection dans des corpus journalistiques
de l'acquisition d'entreprises. En effet, à chaque nouveau système d'extraction
d'information dans un domaine différent, il est nécessaire de redéfinir les
différents patrons sémantiques et d'entraîner un nouveau système sur de
nouvelles données. En s'appuyant sur le corpus FrameNet et pour évoluer vers
une généralisation de ces systèmes, \cite{gildea2002automatic} présentent le
premier système général d'annotation en rôles sémantiques. Sans pour autant
remplacer les systèmes d'extraction d'information, l'annotation en rôles
sémantiques a été utilisée dans diverses applications, notamment les systèmes
de questions-réponses \citep{shen2007using}, l'extraction d'évènements
\citep{exner2011using},  l'analyse d'opinions \citep{das2012structure} ou la
traduction automatique \citep{bazrafshan2013semantic}. Un des intérêts de la
généralité de l'annotation en rôles sémantiques est de ne pas être limitée aux
tâches les plus classiques en Traitement Automatique des Langues. Ainsi,
l'annotation en rôles sémantiques à été utilisée pour la détection de plagiats
\citep{osman2012improved}, la prédiction des cours de bourses
\citep{xie2013semantic}, la génération de scènes 3D \citep{chang2014semantic},
ou l'interprétation de recettes de cuisine \citep{malmaud2014cooking}.

\section{Contraintes}

Afin que le système puisse être utile pour différentes tâches en Traitement
Automatique des Langues, nous souhaitons une analyse sémantique utilisable dans
l'analyseur linguistique libre LIMA \citep{besancon2010lima}. Trois contraintes
majeures en découlent.

\paragraph{Cadre ouvert} Se contenter de désambiguïser certains mots ou se
limiter à un domaine fermé n'est pas satisfaisant ici. Les inventaires de sens
utilisés doivent couvrir l'ensemble des sens présents dans une langue. La
contrainte de traiter l'ensemble des verbes du vocabulaire est ici plus
importante que de pouvoir traiter les noms, adjectifs et adverbes.

\paragraph{Langue française} Le français dispose d'un nombre limité de
ressources sémantiques en cadre ouvert. Il n'existe pas aujourd'hui de VerbNet,
WordNet ou de FrameNet du français avec une couverture et une qualité proche de
leurs équivalents respectifs en langue anglaise. Ne disposant pas des moyens
pour créer de telles ressources manuellement, le système présenté doit pouvoir
se contenter de transpositions automatiques de ces ressources vers le français.

\paragraph{Simplicité} Il n'est pas possible de faire abstraction des réalités
d'un laboratoire qui se situe à l'interface entre la recherche et l'industrie
et dans lequel une partie du personnel peut réfléchir sur des projets sur le
long terme alors qu'une autre partie doit s'assurer que les travaux effectués
soient finis au terme de leur contrat pour être potentiellement utilisés et
repris par d'autres personnes ensuite. La stratégie que nous adoptons pour
pallier ce problème est de simplifier nos systèmes, et des les améliorer dès
qu'ils ont montré leurs limites.

\paragraph{Efficacité} Cette contrainte est moins forte que les autres, mais
reste nécessaire pour que les solutions présentées puissent être utilisées dans
l'analyseur LIMA. L'annotation en rôles sémantiques est un problème difficile
de classification automatique et certains systèmes ont des temps d'entraînement
et d'exécution trop longs pour une utilisation à large échelle.

\section{FrameNet}
\label{presentation_framenet}
% TODO définir FrameNet (peut être une sous-section)

\begin{figure}[ht]
    \centering
    \begin{tabular}{ccc}
        \toprule
        Carol & crushed   & the ice \\
        Agent & V         & Patient \\
        \midrule
        The ice & crushes & easily  \\
        Patient & V       &         \\
        \bottomrule
    \end{tabular}
    \caption{\label{fig:example_srl}Ces deux phrases annotées avec la classe VerbNet carve-21.2 mettent en évidence que la position des arguments ne détermine pas directement les rôles: le sens et la voix de \textit{crush} ne change pas mais l'annotation sémantique est différente.}
\end{figure}


\section{VerbNet}
\label{presentation_verbnet}

Plus de détails sur notre traduction de VerbNet au Chapitre~\ref{ch:verbnet}.
