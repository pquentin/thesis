% vim: set spelllang=fr:
\chapter{Introduction}
\label{ch:intro}

Désambiguïsation lexicale et annotation en rôles sémantiques sont deux tâches
d'analyse sémantique aux applications nombreuses, telles que l'extraction et la
recherche d'information, la traduction automatique, ou encore le résumé
automatique de textes. Des ressources ont été développées pour faciliter le
traitement de chacune de ces tâches, ainsi que différentes approches
statistiques. Nous nous intéressons ici à l'évaluation des performances de ces
deux tâches dans un cadre ouvert et aux possibilités de traitement des langues
moins dotées que l'anglais en ressources sémantiques.

\section{Introduction}

% L'idée d'avoir une phrase sur laquelle examiner toutes les "choses" qu'on
% souhaite décrire est bonne. Qu'est-ce qu'on souhaite décrire ? SRL oui, WSD
% pas trop. Autre chose ?
% C'est pour cette raison que le système n'est pas encore découpé

En Traitement Automatique des Langues, l'analyse sémantique étudie le sens des
énoncés. Nous nous intéressons ici à deux tâches en particulier : la
désambiguïsation lexicale et l'annotation en rôles sémantiques.

\begin{itemize}

  \item \textbf{Désambiguïsation lexicale} Elle a pour but de déterminer le
      sens correct d'un mot dans son contexte. Par exemple, est-ce que le mot «
      mineur » réfère à un travailleur ou à une jeune personne ? Il faut
      d'abord régler la question épineuse de la définition du sens, puis se
      baser sur le contexte proche pour faire le bon choix.

  \item \textbf{Annotation en rôles sémantiques} Elle répond à la question «
      Qui a fait Quoi à Qui, Comment, Où et Quand ? ». Il faut d'abord définir
      quelle est la situation observée, puis déterminer les rôles que jouent
      les différents syntagmes de la phrase. Par exemple, pour une phrase
      décrivant une rencontre, il faudra détecter qu'une rencontre est décrite,
      puis identifier les deux agents qui se rencontrent, mais aussi
      potentiellement l'endroit et l'heure de la rencontre, la façon dont ils
      se sont rencontrés, etc.

\end{itemize}

Prenons pour exemple la phrase suivante : « Mrs. Aouda essaya vainement de
retenir Mr. Fogg. » (extrait du \textit{Tour du monde en quatre-vingts jours},
Jules Verne). La désambiguïsation lexicale aura pour objectif de déterminer le
sens de tous les mots ambigus en s'aidant du contexte. Parmi les mots pleins, «
essaya » est ambigu (elle n'essaie pas un vêtement, mais tente quelque chose),
ainsi que « retenir » (elle ne veut pas se souvenir de quelque chose, mais
empêcher quelqu'un de faire quelque chose). Au contraire, « vainement » n'a
qu'un seul sens.

L'annotation en rôles sémantique déterminera quant à elle que la phrase
correspond à une situation de tentative (grâce à la présence du verbe « essayer
»), puis déterminera quel est l'agent, l'activité qui a été tentée, et le
résultat (« vainement »). Ainsi, le résultat de l'annotation serait :

\begin{figure}[htbl]
    \centering
    \begin{tabular}{cccc}
      [Agent]  & \textbf{Tentative} & [Résultat]  & [Activité]           \tabularnewline
    Mrs. Aouda & \textbf{essaya}  & vainement  & de retenir Mr. Fogg. \tabularnewline
    \end{tabular}
    \caption{Une phrase annotée en rôles sémantiques}
\end{figure}

\subsection{Applications WSD}

La désambiguïsation lexicale a de nombreuses applications potentielles, la
première étant la traduction automatique. C'est en effet l'application
historique, identifiée dès 1949 \citep{weaver1949translation}. Le mot italien «
penna » peut être traduit en français avec les mots « plume », « stylo » et «
pâtes ». C'est donc clairement le sens des mots qui définit la traduction
automatique. Les performances actuelles de la désambiguïsation lexicale ne
permettent pas d'affirmer avec certitude une amélioration pour la traduction
automatique \citep{apidianaki2009place}, mais des travaux encourageants
existent \citep{chan2007word}.

Un autre exemple classique est la recherche d'information où le sens des mots
cherchés peut aider à définir les documents à retourner. En fouille d'opinion,
établir le sens des mots utilisés est aussi important : par exemple, un navet
sera négatif dans le domaine du cinéma, neutre en cuisine. Étant donné la
difficulté à introduire des outils de désambiguïsation lexicale dans des
systèmes existants \citep{navigli2009word}, et dans le but de poursuivre leur
amélioration, ceux-ci sont évalués \textit{in vitro}, comme l'attestent les
tâches présentées lors des dernières campagnes d'évaluation sur la
désambiguïsation lexicale
\citep{agirre2010semeval,manandhar2010semeval,lefever2010semeval}.

\subsection{Applications SRL}

En ce qui concerne l'annotation en rôles sémantiques, elle est issue des
travaux sur l'extraction d'information où les systèmes traitent des situations
très spécifiques, par exemple la détection de résultats d'évènements sportifs
ou la détection dans des corpus journalistiques de l'acquisition d'entreprises.
À chaque nouveau système d'extraction d'information dans un domaine différent,
il est nécessaire de redéfinir les différents patrons sémantiques et
d'entraîner un nouveau système sur de nouvelles données. L'annotation en rôles
sémantiques a été pensée par \cite{gildea2002automatic} comme un moyen
d'évoluer vers une généralisation de ces systèmes. Elle a depuis été utilisée
dans diverses applications, notamment les systèmes de questions-réponses
\citep{shen2007using} et l'analyse d'opinions \citep{das2012structure}.

\subsection{Contraintes (ALL)}

Nous souhaitons une analyse sémantique utilisable dans le logiciel LIMA
\citep{besancon2010lima}. Trois contraintes majeures en découlent.

\paragraph{Cadre ouvert} Se contenter de désambiguïser certains mots ou se
limiter à un domaine fermé n'est pas satisfaisant ici. Les inventaires de sens
utilisés doivent couvrir l'ensemble des sens présents dans une langue.

\paragraph{Langue française} Le français dispose d'un nombre limité de
ressources sémantiques en cadre ouvert. Il n'existe par exemple pas de WordNet
ou de FrameNet français avec une couverture aussi large que celle de leurs
équivalents respectifs en langue anglaise. Ne disposant pas des moyens pour
créer une telle ressource manuellement, il faut transposer de manière
automatique ces ressources vers le français avec pour objectif une couverture
maximale.

\paragraph{Simplicité} Il n'est pas possible de faire abstraction des réalités
d'un laboratoire qui se situe à l'interface entre la recherche et l'industrie
et dans lequel une partie du personnel est amenée à pouvoir faire des projets
sur le long terme alors qu'une autre partie doit s'assurer que les travaux
effectués pourront être repris par d'autres personnes ensuite. La stratégie que
nous adoptons pour pallier ce problème est de considérer la simplicité des
systèmes comme un objectif majeur, qui pourra être repris et amélioré par la
suite.

\paragraph{Efficacité} Cette contrainte est moins forte que les deux autres,
mais reste nécessaire pour que les solutions présentées puissent être
utilisées dans l'analyseur LIMA.  C'est un problème difficile de
classification automatique. En effet, les sens possibles sont différents pour
chaque mot, ce qui nécessite d'entraîner un classifieur sur chaque mot présent
dans le texte. Tous les mots doivent alors disposer d'un grand nombre
d'exemples annotés pour une approche complètement supervisée, ce qui n'est pas
réalisable en pratique. Les techniques que nous présentons ici visent à
contourner ces difficultés.

