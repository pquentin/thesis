% vim: set spelllang=fr:
\chapter{Introduction}
\label{ch:intro}

\section{L'annotation en rôles sémantiques}

L'annotation en rôles sémantiques est une tâche d'analyse sémantique aux
applications nombreuses, telles que l'extraction et la recherche d'information,
la traduction automatique, ou encore le résumé automatique de textes.

% L'idée d'avoir une phrase sur laquelle examiner toutes les "choses" qu'on
% souhaite décrire est bonne. Qu'est-ce qu'on souhaite décrire ? SRL oui, WSD
% pas trop. Autre chose ?
% C'est pour cette raison que cette section n'est pas encore découpée

% TODO utiliser exemple stage Guilhem ?
Elle répond à la question « Qui a fait Quoi à Qui, Comment, Où et Quand ? ».
Il faut d'abord définir quelle est la situation observée, puis déterminer les
rôles que jouent les différents syntagmes de la phrase. Par exemple, pour une
phrase décrivant une rencontre, il faudra détecter qu'une rencontre est
décrite, puis identifier les deux agents qui se rencontrent, mais aussi
potentiellement l'endroit et l'heure de la rencontre, la façon dont ils se sont
rencontrés, etc.

Prenons pour exemple \emph{Mrs. Aouda essaya vainement de retenir Mr. Fogg}
(extrait du \emph{Tour du monde en quatre-vingts jours} de Jules Verne).
L'annotation en rôles sémantique déterminera que la phrase correspond à une
situation de tentative (grâce à la présence du verbe « essayer »), puis
déterminera quel est l'agent, l'activité qui a été tentée, et le résultat («
vainement »). Ainsi, le résultat de l'annotation serait :

\begin{figure}[htbl]
    \centering
    \begin{tabular}{cccc}
      [Agent]  & \textbf{Tentative} & [Résultat]  & [Activité]           \tabularnewline
    Mrs. Aouda & \textbf{essaya}  & vainement  & de retenir Mr. Fogg. \tabularnewline
    \end{tabular}
    \caption{Une phrase annotée en rôles sémantiques}
\end{figure}

\section{Applications}

% Ici on répète gildea&jurafsky, le rendre plus explicite
L'annotation en rôles sémantiques est une évolution naturelle de certains
travaux sur l'extraction d'information où les systèmes traitent des situations
très spécifiques, par exemple la détection de résultats d'évènements sportifs
ou la détection dans des corpus journalistiques de l'acquisition d'entreprises.
En effet, À chaque nouveau système d'extraction d'information dans un domaine
différent, il est nécessaire de redéfinir les différents patrons sémantiques et
d'entraîner un nouveau système sur de nouvelles données. L'annotation en rôles
sémantiques a été pensée par \cite{gildea2002automatic} comme un moyen
d'évoluer vers une généralisation de ces systèmes. Elle a depuis été utilisée
dans diverses applications, notamment les systèmes de questions-réponses
\citep{shen2007using} et l'analyse d'opinions \citep{das2012structure}.

\section{Contraintes}

Nous souhaitons une analyse sémantique utilisable dans le logiciel LIMA
\citep{besancon2010lima}. Trois contraintes majeures en découlent.

\paragraph{Cadre ouvert} Se contenter de désambiguïser certains mots ou se
limiter à un domaine fermé n'est pas satisfaisant ici. Les inventaires de sens
utilisés doivent couvrir l'ensemble des sens présents dans une langue.

\paragraph{Langue française} Le français dispose d'un nombre limité de
ressources sémantiques en cadre ouvert. Il n'existe par exemple pas de WordNet
ou de FrameNet français avec une couverture aussi large que celle de leurs
équivalents respectifs en langue anglaise. Ne disposant pas des moyens pour
créer une telle ressource manuellement, il faut transposer de manière
automatique ces ressources vers le français avec pour objectif une couverture
maximale.

\paragraph{Simplicité} Il n'est pas possible de faire abstraction des réalités
d'un laboratoire qui se situe à l'interface entre la recherche et l'industrie
et dans lequel une partie du personnel est amenée à pouvoir faire des projets
sur le long terme alors qu'une autre partie doit s'assurer que les travaux
effectués pourront être repris par d'autres personnes ensuite. La stratégie que
nous adoptons pour pallier ce problème est de considérer la simplicité des
systèmes comme un objectif majeur, qui pourra être repris et amélioré par la
suite.

\paragraph{Efficacité} Cette contrainte est moins forte que les deux autres,
mais reste nécessaire pour que les solutions présentées puissent être
utilisées dans l'analyseur LIMA.  C'est un problème difficile de
classification automatique. En effet, les sens possibles sont différents pour
chaque mot, ce qui nécessite d'entraîner un classifieur sur chaque mot présent
dans le texte. Tous les mots doivent alors disposer d'un grand nombre
d'exemples annotés pour une approche complètement supervisée, ce qui n'est pas
réalisable en pratique. Les techniques que nous présentons ici visent à
contourner ces difficultés.
