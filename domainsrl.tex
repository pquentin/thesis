\chapter{Annotation en rôle sémantique basée sur la connaissance en domaine spécifique}
\label{ch:domainsrl}

Là où le précédent chapitre présentait sur l'annotation en rôles sémantiques
basée sur la connaissance, celui-là se concentre sur l'annotation en domaines
spécifiques.

La définition du terme 'domaine' reste assez vague dans le sens où il est
difficile de proposer une catégorisation de la connaissance en un ensemble de
domaines qui soit à la fois cohérente et efficace pour le Traitement Automatique
des Langues. Il est aussi difficile de séparer clairement le «~domaine
général~» des domaines spécifiques.

Néanmoins, c'est un phénomène qui existe et qui est à considérer : les modèles
entraînés sur un corpus d'un domaine spécifique (la finance par exemple)
généraliseront mal à d'autres domaines (le football) par exemple. À l'heure où
les algorithmes supervisés obtiennent d'excellentes performances sur un certain
nombre de tâche, l'adaptation au domaine est un défi majeur du Traitement
Automatique des Langues.

Il est important aussi de différencier le genre et le domaine d'un texte. Le
corpus du Wall Street Journal traite du domaine de la finance dans un genre
journalistique. D'autres genres existent, par exemple les genre littéraires que
sont la fiction, la poésie, le théâtre, etc. Néanmoins, ces distinctions ne
nous concernent pas ici, et l'objectif affiché est de traiter aussi bien un
roman qu'une encyclopédie qu'un email bien écrit qu'un article de journal.
C'est une simplification mais les difficultés que nous rencontrons avec les
corpus utilisés dépendent plutôt du domaine que du genre. En effet, dans deux
domaines différents plus que dans deux genres différents, ce sont les verbes
utilisés, leur sens et la façon de les utiliser qui change et qui doit être
prise en compte.

\section{Corpus considérés}

Afin de s'assurer de ne pas tomber dans les mêmes travers que les études ne
portant que sur un domaine précis, nous considérons ici trois corpus présents
dans des domaines différents :

\begin{itemize}
    \item le Kicktionary rassemblant des dépêches de l'UEFA dans le domaine du football
    \item le corpus Informatique/Internet de l'OLST
    \item le corpus Réchauffement climatique de l'OLST
\end{itemize}

Les points communs majeurs de ces trois corpus sont :
\begin{enumerate}
    \item de s'inspirer librement de la théorie des Frame Semantics telle qu'elle est considérée dans FrameNet,
    \item d'être disponible à la fois en anglais et en français,
    \item d'être basés sur l'annotation dans un corpus d'un certain nombre de prédicats (ce n'est pas une annotation dite "full-text")
\end{enumerate}

\section{Enrichissement de VerbNet}

Néanmoins, ces corpus posent des problèmes différents vis-à-vis de leur
annotation avec VerbNet, tous liés au fait de ne pas être dans le "domaine
général". Nous avons considéré comme problème les manquements dans VerbNet qui
empêchent de prendre en compte les phrases considérées, indépendamment de
l'algorithme. Ainsi, pour que VerbNet puisse prendre en compte une instance, il faut :

\begin{itemize}
    \item que le sens du verbe considéré soit présent dans VerbNet,
    \item et que la construction en question soit présente dans VerbNet.
\end{itemize}

Il y a différentes façons de ne pas respecter ces deux contraintes :

\begin{itemize}
    \item Le verbe n'existe pas du tout dans VerbNet
    \item Le sens du verbe utilisé n'est pas représenté dans VerbNet alors que c'est un sens du domaine général
    \item Le sens du verbe utilisé n'est pas représenté dans VerbNet alors que c'est un sens spécifique à ce domaine
    \item La construction utilisée n'est pas présente dans VerbNet
\end{itemize}

Suivant les corpus spécifiques à un domaine, les proportions d'erreurs seront
différentes. Nous avons demandé à deux annotateurs d'évaluer, pour vingt
phrases par domaine, quelle était l'erreur. L'accord inter-annotateur permet de
valider la distinction domaine spécifique/domaine général malgré
l'impossibilité de la définir précisément.

C'est pour cette raison que l'enrichissement de VerbNet se fait de deux
manières : certains verbes et constructions sont ajoutés comme faisant partie
du domaine général, alors que d'autres sont étiquetés avec le domaine
spécifique correspondant. L'idée est d'éviter que des connaissances de domaines
spécifiques viennent réduire la qualité de la ressource tout en s'assurant que
la couverture de VerbNet pour le domaine général continue de s'améliorer.

\subsection{Détection semi-automatique d'erreurs}

D'une part, pour minimiser le travail manuel, il est important d'automatiser au
maximum l'enrichissement de la ressource. D'autre part, l'intérêt de VerbNet
réside notamment dans sa capacité à factoriser efficacement des informations
syntaxico-sémantiques précises sur les verbes d'une langue, et il est donc
important de valider manuellement chacun des changements apportés.

Le principe suivi est donc d'essayer de détecter les manquements de VerbNet de
manière automatique avant de proposer à un utilisateur expert de réaliser des
changements en ayant un maximum d'informations pertinentes à portée de main.

Les différentes erreurs détectées sont :
\begin{itemize}
    \item l'absence de lemmes dans la ressource
    \item l'absence d'un sens correct dans la ressource
    \item l'absence de constructions correctes dans la ressource
\end{itemize}

Nous avons d'abord commencé par détecter l'absence de constructions correctes,
ce qui a permis ensuite de détecter une absence de sens en comparant les
constructions observées avec les constructions présentes. Enfin, cela a permis
de faire des propositions quant à la position d'un nouveau lexème dans la
ressource.

\section{Évaluation}

Les trois corpus considérés sont annotés en rôles sémantiques dans les deux
langues considérées : l'anglais et le français. Il nous suffit donc d'utiliser
un mapping entre les rôles considérés par ces ressources et les rôles VerbNet
avant d'évaluer la performance de l'algorithme basé sur la connaissance
d'annotation en rôles sémantiques présenté au chapitre~\ref{ch:srl}.
