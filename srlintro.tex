% vim: set spelllang=fr:
\chapter{Introduction}
\label{ch:srlintro}


Computing automatically a semantic representation of a given text has been
defined as the single greatest limitation on the general application of natural
language processing techniques \citep{dang1998investigating}. Semantic role
labeling provides such a representation where chunks of a sentence are
annotated with semantic roles that denote the sense of those chunks related to
the verb. In this work we use VerbNet \citep{kipperschuler2005verbnet} and give
some details about it later. Figure~\ref{fig:example_srl} shows an example
highlighting the difficulty of the task which can not rely exclusively on
syntactic clues but also needs semantic knowledge.

\begin{figure}[ht]
    \centering
    \begin{tabular}{ccc}
        \toprule
        Carol & crushed   & the ice \\
        Agent & V         & Patient \\
        \midrule
        The ice & crushes & easily  \\
        Patient & V       &         \\
        \bottomrule
    \end{tabular}
    \caption{\label{fig:example_srl}Ces deux phrases annotées avec la classe VerbNet carve-21.2 mettent en évidence que la position des arguments ne détermine pas directement les rôles: le sens et la voix de \textit{crush} ne change pas mais l'annotation sémantique est différente.}
\end{figure}

Semantic role labeling has tremendously helped to compute semantic
representations and has been shown to improve multiple tasks such as
information extraction \citep{surdeanu2003using}, question-answering
\citep{shen2007using}, event extraction \citep{exner2011using},
plagarism detection \citep{osman2012improved}, machine translation
\citep{bazrafshan2013semantic} or even stock price prediction
\citep{xie2013semantic}.

