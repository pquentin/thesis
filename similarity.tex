% vim: set spelllang=fr:
\chapter{Informations sémantiques pour l'annotation en rôles sémantiques}
\label{ch:semantic}

Deux types de données aident à l'annotation en rôles sémantiques : d'une part
les informations syntaxiques, et d'autre part les informations sémantiques.
Nous avons traité au chapitre~\ref{ch:srl} la question de l'apport de la
syntaxe à l'annotation en rôles sémantiques avec l'utilisation de VerbNet. Il
est par exemple probable que le sujet syntaxique d'une phrase soit Agent plutôt
que Thème.  Dans ce chapitre, nous nous intéressons à l'apport sémantique des
différents arguments du verbe. En effet, c'est une information qui est utile
aussi pour annoter un prédicat en rôles sémantiques.

Prenons pour exemple ces cinq phrases mettant en jeu le verbe
\emph{conspire} de la classe \texttt{conspire-71} du VerbNet anglais :

\begin{itemize}
    \item \emph{Alan Rock conspired with passionate scientists like Henry Friesen}
    \item \emph{P. Baglioni conspired against the Borgias}
    \item \emph{He joined Mazzini and conspired for the redemption of Italy}
    \item \emph{The companies conspire together on price-fixing}
    \item \emph{The delegation organized by Mr. Chan-Ou-Teung did not conspire about holding a UBCV Congress}
\end{itemize}

L'annotation du sujet syntaxique ne pose pas de difficultés : le sujet est
toujours Agent dans cette classe selon VerbNet. Les informations syntaxiques
présentes dans VerbNet suffisent aussi à annoter les deux premières phrases:
\emph{with passionate scientists like Henry Friesen} est Co-Agent alors que
\emph{against the Borgias} est Beneficiary. Ce n'est pas le cas des phrases
suivantes : les prépositions \emph{for}, \emph{on} et \emph{about} ne sont pas
prévues pas VerbNet.

Ici, une solution est de continuer à agrandir VerbNet en incluant de nouvelles
constructions ou en étendant les constructions existantes, mais c'est un
travail long et difficile qui ne pourra vraisemblablement jamais être terminé.

% TODO mieux analyser ces cas pour identifier les causes
De plus, pour 25\% des arguments de notre corpus FrameNet, la syntaxe ne permet
pas d'identifier le rôle correct, mais seulement de le limiter à quelques rôles
possibles. Cette délimitation est très précise : quand VerbNet aboutit à
plusieurs rôles possibles, la probabilité pour que le rôle correct soit présent
dans la liste est supérieure à 90\%. Nous posons alors la question suivante :
comment tirer profit de cette courte liste de rôles possibles à l'aide de la
sémantique d'autre part ?

La première piste, correspondant à la première section de ce chapitre, est
d'utiliser les restrictions de sélections présentes dans VerbNet. Ces
restrictions sont relativement grossières et mal documentées mais sont une
première piste explorées dans la section~\ref{sec:restr_verbnet}.

La seconde piste (section~\ref{sec:similarity_module} consiste à apprendre de
manière supervisée des scores de similarité entre un rôle et un remplisseur de
rôle potentiel. Cette piste a été explorée dans le cadre de notre participation
au projet ANR ASFALDA dont le but est de produire un FrameNet du français, un
analyseur automatique et un exemple d'application de l'analyseur.

\section{Restrictions de sélection VerbNet et WordNet}
\label{sec:restr_verbnet}

En travail futur, nous aimerions ici disposer de restriction de sélections à la
fois fines mais disposant aussi d'une bonne couverture sur l'ensemble des
verbes du vocabulaire.

\section{Apprentissage supervisé de la similarité sémantique entre un remplisseur et un rôle}
\label{sec:similarity_module}
