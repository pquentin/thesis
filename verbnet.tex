% vim: set spelllang=fr:
\chapter{Traduction de VerbNet pour le français}
\label{ch:verbnet}

\section{Introduction}

% TAL a besoin d'infos sur les verbes, c'est difficile et fragile, et on a
% besoin d'énormes corpus
Le Traitement Automatique des Langues requiert des lexiques et de larges
quantités de données annotées pour analyser efficacement des textes dans le
domaine général. Obtenir cette quantité de données est un problème en soi connu
sous le nom de "knowledge acquisition bottleneck" en désambiguïsation lexicale
\citep{gale1992using}. Annoter de plus en plus de données de manière efficace
réduira le bottleneck pour certains domaines, mais d'autres stratégies sont
nécessaires pour atteindre nos objectifs. Une possibilité est d'utiliser au
mieux les données annotées en perfectionnant les algorithmes existants, une
autre est d'utiliser intelligemment les données non annotées qui existent en
quantité bien plus importante. Une troisième possibilité selon nous est
d'encoder les lexiques de manière efficiente pour couvrir une large partie du
vocabulaire. C'est ce qui est fait dans VerbNet où les traits partagés par les
mêmes verbes sont explicitement notés, ce qui permet à chaque modification dans
VerbNet de mieux traiter un certain nombre de verbes au lieu d'un seul.

% VerbNet a montré son efficacité dans le domaine de par son approche
% pragmatique. Vient des classes de Levin, donne des correspondances entre
% syntaxe et sémantique
Deux difficultés majeures qu'affrontent les créateurs de lexique sont la
granularité de sens et la distinction des sens. Ces deux difficultés sont
gérées par les classes de Levin \citep{levin1993english}. Dans ces classes, les
verbes sont classifiés principalement à travers leur alternances syntaxiques,
ce qui fournit un critère qui est à la fois facilement observable et qui
produit des distinctions sémantiques intéressantes. VerbNet
\citep{kipperschuler2005verbnet} est un lexique électronique basé sur les
classes de Levin. Il encode non plus les alternances mais les cadres de
sous-catégorisation valables pour chaque classe, et rajoute des informations de
rôle et de sémantique à travers une logique des prédicats simplifiée. De
nouvelles classes, constructions et verbes ont étés ajoutés à VerbNet au fil
des ans.

% VN est bon pour SRL
VerbNet est un lexique adapté au Traitement Automatique des Langues : on peut
utiliser un cadre de sous-catégorisation pour mapper des syntagmes à des rôles
thématiques \citep{swier2005exploiting,pradet2013revisiting}. Grâce à sa
couverture élevées (plus de quatre mille verbes distincts) et son groupement de
verbes utile, VerbNet est bien adapté à l'annotation en rôles sémantiques.

% Dans les langues autres que l'anglais, cette ressource utile n'existe pas
% mais ne demande qu'à exister étant donné le potentiel \textit{cross-lingual}
% de VerbNet. Il y a souvent des ressources proches, moins utiles, plus
% linguistiques, existent.
Cependant, un VerbNet de qualité n'existe pour le moment que pour l'anglais.
Une telle ressource serait pourtant encore plus utile pour les langues moins
ressourcées où les corpus annotés en rôles sémantiques n'existent pas. VerbNet
a un potentiel inter-linguistique, visible notamment avec le portuguais
\citep[section 2.2.2]{kipperschuler2005verbnet}. Adapter VerbNet vers une
nouvelle langue suffisamment proche de l'anglais permet de conserver sa
structure, ainsi que l'information sémantique et les rôles thématiques, ce qui
donne la possibilité de produire un lexique utile sans des années de travail
manuel.

% En particulier, en français, les tables du LADL décrivent le comportement
% syntaxique des verbes et le LVF découpe les verbes du français
% sémantiquement. Un certain nombre de difficultés empêche d'utiliser ces
% ressources en TAL

En ayant pour but de développer une version français de VerbNet (nommée
\verbenet{}), nous avons d'abord (section~\ref{first}) traduit vers le français
les membres des classes de premier niveau dans VerbNet en utilisant des
ressources linguistiques qui encodent des informations syntaxiques et
sémantiques sur les verbes du français. La deuxième étape
(section~\ref{second}), toujours en cours, est d'adapter les cadres de
sous-catégorisation vers le français et de réorganiser les classes de VerbNet
en fonction, ce qui pose divers problèmes que nous décrirons en partie ici. La
troisième étape (section~\ref{third}) sera l'occasion de valider les membres
français de chaque classe.

Notre traduction de VerbNet sera liée aux deux ressources linguistiques que
nous utilisons, Les Verbes Français et le Lexique-Grammaire. La ressources sera
aussi ouverte : nous voulons encourager les cosntributions externes avec notre
outil basé sur le web. Nous souhaitons aussi faciliter l'utilisation de la
ressource en utilisant le même format XML que le VerbNet anglais.

\section{Présentation de VerbNet et des ressources lexicales françaises}
\label{french}

\subsection{Les classes de Levin}

% Bien distinguer alternances vs. cadres de sous-cat

\cite{levin1993english} est une classification des verbes anglais suivant un
principe simple : le comportement syntaxique des verbes détermine en partie
leur sens. Après avoir défini un certain nombre d'alternances de diathèses
possibles, les verbes sont classés en groupes partageant les mêmes
constructions.

TODO exemple d'alternance venant vraiment de Levin.

Cette classification a trois intérêts pour l'annotation en rôles sémantiques :

\begin{itemize}

    \item Une grande majorité des verbes anglais sont couverts, rendant la
        ressource utile pour des annotations à large échelle.

    \item La classification est hiérarchique et regroupe de nombreux verbes :
        avec une cinquantaine de classes principales, des généralisations et
        partages d'informations entre les verbes de classes identiques ou
        proches sont possibles.

    \item Les comportements syntaxiques déterminent les comportements
        sémantiques, ce qui correspond au schéma classique de l'annotation en
        rôles sémantiques qui s'appuie sur une analyse syntaxique.

\end{itemize}

%\subsection{Intersection des classes de Levin}

% TODO raccourcir sans dire de bêtises

Les classes de Levin ne s'intéressent pas directement au sens, et regroupent
parfois des verbes dont le sens peut être très différent.  Ainsi, dans les
\textit{put verbs} se trouve la classe \textit{Pocket} qui regroupe les verbes
"mettre dans sa poche" et "mettre en prison". Cependant, le comportement
syntaxique est le même : le regroupement est logique et les différences de sens
ne sont a priori pas gênantes pour nos applications qui ne se basent pas sur le
sens des verbes dans VerbNet.

Pour affiner le sens des classes obtenues, \cite{dang1998investigating} propose
les classes de Levin intersectives en donnant l'exemple d'intersections de
classes qui donnent des ensembles sémantiquement cohérents. Ainsi, les
\textit{split verbs} sont définis comme des verbes séparant par une action en
\textit{-ing}. Ces verbes incluent \textit{cut}, \textit{rip}, \textit{split}
mais aussi \textit{push}, \textit{kick} ou encore \textit{slip}, étant donné
que \textit{push} par exemple sépare de par l'action de pousser. Un autre
exemple : les \textit{carry verbs} sont définis comme les verbes causant un
mouvemant, avec une séparation entre les mouvements accompagnés
(\textit{carry}, \textit{drag}, ...) et le mouvement non accompagné
(\textit{shove}, \textit{drag}, ...). Au-delà de cette distinction, c'est une
classe qui contient deux types de verbes : ceux qui acceptent une
\textit{conative alternation} indiquant que le mouvement est essayé sans être
réussi et ceux qui impliquent seulement que le mouvement est causé par l'agent.

En combinant les verbes apparaissant à la fois dans les \textit{split verbs} et
les \textit{carry verbs}, on obtient un ensemble sémantiquement cohérent :
uniquement les verbes indiquant l'application de la force du côté de
\textit{pull}, et uniquement les verbes partageant la \textit{conative
alternation} pour \textit{carry}. Cette opération peut être automatisée : dès
que deux classes partagent au moins trois verbes, \cite{dang1998investigating}
a jugé intéressant de créer une nouvelle classe contenant l'intersection qui
sera cohérente sémantiquement et collera mieux avec la ressource sémantique
WordNet.

% TODO exemple et intérêt pour nous : tirer des exemples de plusieurs classes ?
% Restreindre les frames possibles ?

\subsection{VerbNet}

VerbNet est une extension modernes des classes de Levin, qui se retrouvent
améliorées sur plusieurs fronts :

\begin{itemize}

    % TODO vraiment ?
    \item la couverture des verbes existants a été testée sur PropBank, ce qui
        a mené à l'ajout de nouvelles constructions syntaxiques

    \item de nouvelles classes provenant de \cite{korhonen2004extended}
        intégrant les verbes acceptant des complétives, mais aussi des
        syntagmes adjectivaux et adverbiaux ou encore des particules.

    \item de nouveaux verbes provenant de \cite{dorr2001lcs} ont étés intégrés

    \item les verbes ont étés liés à WordNet, OntoNotes, PropBank et FrameNet
        \citep{palmer2009semlink}

    \item de nombreuses corrections ont étés apportées au fil des versions.

\end{itemize}

Ces améliorations ont à la fois contribué à VerbNet en largeur (nouvelles
classes) et en profondeur (nouveaux verbes, nouvelles constructions
syntaxiques).

% TODO nombre de sous-classes
% TODO nombre de verbes
La hiérarchie supérieure de VerbNet contient 270 classes. Certaines de ces
classes sont sous-divisées pour former une hiérarchie relativement plate :
VerbNet contient en tout XXX classes. Pour chaque (sous-)classe, ce lexique
indique :

\begin{itemize}
        \item la liste des verbes de la classe,
        \item les rôles thématiques en jeu ainsi que leur restrictions de sélection,
        \item la liste des \emph{frames} VerbNet.
\end{itemize}

Une frame inclut une phrase d'exemple, une formule syntaxique donnant la
liaison entre les syntagmes et les rôles thématiques, une formule sémantique
basée sur la logique des prédicats explicitant la relation entre les
participants et les évènements.

% TODO ?

\subsection{Les Verbes Français et Lexique-Grammaire}

Un des intérêts des classes de Levin et de VerbNet par rapport aux ressources
françaises est leur approche pragmatique qui se traduit notamment par l'absence
de prise en compte des emplois métaphorique d'un verbe donné. Ainsi, les tables
du LADL incluent des usages tels que \textit{L'idée gallopait dans son esprit}
qui peuvent induire en erreur une application. \cite{brown2012semantic}
proposent une analyse systématique des emplois métaphorique de deux verbes
représentatifs et montrent qu'utiliser VerbNet pour raisonner sur les emplois
métaphorique d'un texte est en partie possible au prix d'une complexité plus
importante et de prédicats sémantiques moins précis.

À partir des années 70 deux ressources lexicales pour les verbes français ont
été dévelopées : LVF et LG. \footnote{Plus tard, dans les années 1990, une
autre ressource a été dévelopée : Dicovalence. Nous ne l'utilisons presque pas
dans nos travaux.}

\begin{itemize}

    % TODO mieux que sous-sous-sous et nombre de E2f.1
    \item LVF (Les Verbes Français, \cite{dubois1997verbes}) contient environ
        25000 entrées classées en 14 classes sémantiques, 54 sous-classes
        syntactico-sémantiques, 248 sous-sous-classes et XXX
        sous-sous-sous-classes.

    \item LG (Lexique-Grammaire, \cite{gross1975methodes,boons1976structure})
        comporte lui 14 000 entrées classifiées en 67 «~tables~», chaque table
        groupant des verbes partageant la même propriété définitoire syntaxique
        et potentiellement une sémantique similaire. Chaque colonne de la table
        encode des restrictions supplémentaires s'appliquant à certains des
        verbes de la table.

    %\item DICOVALENCE \citep{} est basé sur l'approche pronominale \citep{} :
    %    ce lexique liste 5 000 entrées parmi les plus fréquentes en français.

\end{itemize}

Les classes LVF et les tables LG peuvent toutes les deux être comparées aux
classes VerbNet. Cependant, ces (vieilles) ressources n'encodent ni les rôles
thématiques ni les formules sémantiques \footnote{Les notions de rôles
thématiques et d'évènement n'étaient pas répandues dans les années 1970.}.
C'est la raison pour laquelle nous voulons construire une nouvelle ressource
française, \verbenet{}\footnote{Le nom vient de la prononciation à la française
qui "rajoute" un e. Le $\ni$ est simplement là pour bien marquer ce e et ainsi
essayer d'éviter la confusion avec VerbNet}. Cette ressource tire profit d'une
part des ressources existantes pour le français avec un encodage sémantique et
syntaxique riche, et d'autre part de l'information sémantique présente dans
VerbNet pour l'anglais, une langue proche du français.

\section{Construction de \verbenet{}}

% TODO zero-related to a noun = dénominal ?
Notre principe de base est que la hiérarchie supérieure de VerbNet doit être
aussi proche que possible de celle de VerbNet et ses 270 classes. Néanmoins,
certaines classes peuvent disparaître. Ceci peut être dû à des raisons purement
morphologiques. Une classe VerbNet ne contenant que des verbes dénominaux n'a
pas d'équivalent en français. C'est le cas de \texttt{pit-10.7} avec des verbes
tels que \emph{bark} et \emph{bone} ou \texttt{week-end-56} et ses verbes
\emph{week-end} ou \emph{december}. Par contre, \texttt{debone-10.8} et ses
verbes formés par le préfixe \emph{dé-} plus un nominal (\emph{debark, debone})
a un équivalent français avec les préfixes \emph{dé-} ou \emph{é}
(\emph{déveiner}, \emph{équeuter}).

Étant donné ce principe de bases, la construction de \verbenet{} se fait en
trois étapes.

\subsection{Première étape}\label{first}

La première étape pour construire \verbenet{} est de déterminer quels verbes
français appartiennent à une des 270 classes de VerbNet. Ceci a été fait en
trois étapes.

\begin{enumerate}

    \item Pour une classe VerbNet donnée \Ce{}, nous assignons manuellement les
        classes LVF \Clvf{} et les classes LG \Clg{} qui correspondent à sa
        définition sémantique : nous assignons par exemple {\color{red}L3b} et
        {\color{green}38LD} à {\color{blue}put-9.1} et assignons
        {\color{red}M1a} et {\color{green}32CL ou 32R3 ou 32C}) à
        {\color{blue}body\_internal\_motion-49}.

    \item Nous utilisons deux dictionnaires bilingues (SCI-FRAND-EURADIC et le
        Wiktionnaire) qui fournissent la liste \Ltrad{} des traductions
        françaises des verbes anglais appartenant à \Ce{} et ses sous-classes.

    \item Les verbes français de la classe \Ce{} sont alors simplement les
        verbes appartenant à la fois à \Ltrad{}, \Clvf{} et
        \Clg{}\footnote{Quand cette intersection est vide, c'est la liste
        non-vide (soit \Clvf{} soit \Clg{}) qui est choisie}. Par exemple,
        \emph{mettre}, \emph{poser} et \emph{installer} sont des verbes
        français de {\color{blue}put-9.1}).

\end{enumerate}

Cette étape a été exécutée rapidement et a donné des résultats encourageants :
en ne gardant que les verbes à l'intersection de \Ltrad{}, \Clvf{} et \Clg{},
les résulstats sont précis, et cohérents d'un point de vue sémantique et
syntaxique. Par exemple, la classe {\color{blue}scribble-25.2} contient 18
verbes en anglais : elle est associée à la classe LVF {\color{red}R3a.1} et à
la table LG {\color{green}32A}, ce qui conduit à une liste de 16 verbes
français : \emph{composer}, \emph{couper}, \emph{donner}, \emph{exécuter},
\emph{fabriquer}, \emph{faire}, \emph{forger}, \emph{former}, \emph{imprimer},
\emph{lever}, \emph{produire}, \emph{reproduire}, \emph{sculpter},
\emph{tailler}, \emph{tirer} et \emph{tracer}. Tous ces verbes sont valides
pour cette classe.  Cette méthode produit un lexique de 4058 verbes (2128
verbes distincts).

% TODO An evaluation on 10 randomly chosen classes shows that XX.X\% of verbs are indeed valid.


\subsection{Deuxième étape}\label{second}

% TODO stats sur un VerbNet importé, pas sur VerbeNet à moitié fini
La deuxième étape de la construction de \verbenet{} s'est avérée beaucoup plus
fastidieuse que la première. Pour chacune des 500 classes et sous-classes, nous
déterminons quand cela est possible :

\begin{itemize}

    \item les frames valides pour le français avec des ajustements possibles
        pour les rôles thématiques et les restrictions de sélection,

    \item les sous-classes en français en réorganisant si besoin la hiérarchie
        de l'anglais dans le but d'assigner les verbes obtenus à l'étape 1 à
        une des sous-classes.

\end{itemize}

Cette étape a demandé le développement d'un outil d'édition
(section~\ref{toolquentin}) pour assister le travail lexicographique. Ensuite,
il a fallu définir des principes de base sur les frames françaises quand elle
diffèrent des frames anglaises (section~\ref{princp}). Finalement, une étude
fine au cas par cas révèle des différences difficiles à traiter entre l'anglais
et le français (section~\ref{subsubsec:casebycase}).

\subsubsection{Outil d'édition de \verbenet{}}\label{toolquentin}

Nous avons du développer un outil permettant d'éditer collaborativement
VerbNet. Cet outil est en fait un site web présentant VerbNet : ses classes,
ses frames, ses verbes, etc. Tous les éléments sont modifiables, ce qui permet
aux lexicographes de se concentrer sur les problèmes lexicographiques sans
avoir à se préoccuper de la représentation des données. Le site web est réalisé
avec le framework web Python Django. VerbNet est stocké dans une base de donnée
PostgreSQL. Avant de commencer l'édition, nous avons commencé par charger
VerbNet et les mappings réalisés lors de la première étape. L'édition elle-même
se fait par classe de Levin. Par exemple, nous avons traité les classes
{\color{blue}throw-17.1} et {\color{blue}pelt-17.2} parce qu'elles font toutes
les deux partie de la «~super-classe~» 17 : Throwing. Plusieurs classes faisant
partie d'une même super-classe sont souvent liées et il faut les comprendre
dans leur ensemble avant de commencer l'édition. Pour chaque classes, nous
pouvons éditer ses frames, ajouter ou supprimer des sous-classes ou des frames.
Par exemple, toutes les frames impliquant un conatif, un datif ou une
alternance bénéfactive (?) peuvent être systématiquement supprimées : ces
alternances n'existent pas en français.

% TODO faire une nouvelle capture
\begin{figure}
    \centering
    \includegraphics[width=0.48\textwidth]{fig/tool_screenshot.png}
    \caption{\label{tool}Interface web pour analyser et modifier \verbenet{}.
    Chaque frame peut être modifiée et les classes peuvent être réorganisées. Les
    traductions en violet apaprtiennent à l'intersection de \Clvf{}, \Clg{} et
    \Ltrad{} (section~\ref{first}), les traductions en rouge (respectivement en
    vert) uniquement à \Clvf{} et \Ltrad (respectivement uniquement à \Clg{} et
    \Ltrad{}).}
\end{figure}

Avec l'aide de cet outil (illustré à la Figure~\ref{tool}), la deuxième étape
peut s'avérer très simple. Par exemple, les quatre sous-classes de
{\color{blue}imaeg-creation-25} ont des classes directement équivalentes en
français, donc les seules choses à faire sont de traduire les exemples en
français avec les bonnes prépositions. Par exemple, dans la classe
{\color{blue}illustrate-25.3}, il a fallu remplacer \emph{with} par la
combinaison de \emph{de} et \emph{avec}.

\subsubsection{Principes sur les frames}\label{princp}

Nous avons jusqu'ici identifié deux différences principales entre le codage des
frames anglaises et françaises.

La première différence concerne les sous-structures, c'est-à-dire les frames
avec un complément manquant tel que \emph{NP V} (Luc gravait) dans
{\color{blue}image-impression-25.1}. C'est en effet ici une sous-structure de
\emph{NP V NP.Destination} (Luc gravait les annaux). Le codage de telles
sous-structures est difficile à justifier quand il est basé sur l'introspection
linguistique et nécessite une étude de corpus. Nous ne savons pas comment ce
codage a été fait dans VerbNet et n'avons pas à notre disposition de corpus
français permettant de répondre à la question. Nous avons donc décidé pour le
moment de supprimer toutes les sous-structures de \verbenet{}. Par exemple,
dans la classe {\color{blue}remove-10.1}, VerbNet encode non seulement NP V NP
PP.Source PP.Destination (\emph{Doug removed the smudges from the tabletop})
mais aussi NP V NP (\emph{Doug removed the smudges}). \verbenet{} n'inclut que
la première frame, il est implicite que la seconde existe : une application
doit donc l'inférer automatiquement à partir de la première sans intervention
manuelle\footnote{Cependant, ce principe ne s'applique pas aux verbes acceptant
un seul complément locatif double «~from here to there (un seul
complément PP.Source PP.Destination)~» sans accepter un seul complément source
(PP.Source), tout en acceptant un seul complément destination
(PP.Destination) : \emph{Fred a transferré le vin de la cruche en pierre vers
la cruche en terre cuite, *Fred a transferré le vin de la cruche en pierre,
Fred a transferré le vin vers la cruche en terre cuite}. Dans ces cas, les
sous-structures sont codées}.

La deuxième différence concerne l'ordre des compléments. VerbNet encode parfois
deux frames qui ne diffèrent que par l'ordre des compléments, par exemple dans
{\color{blue}bring-11.3} les frames NP V NP PP.Destination NP (\emph{Nora
brought to lunch the book}) et NP V PP.Destination NP (\emph{Nora brought to
lunch the book}). En français, l'ordre des compléments dépend d'un certain
nombre de facteurs syntaxiques et sémantiques \citep{thuilier2012contraintes},
mais ne dépend pas a priori d'un facteur lexical : il ne dépend pas du verbe
qui gouverne les compléments. C'est pour cette raison que \verbenet{} n'encode
qu'un frame dans ces cas, ici seulement \emph{NP V NP PP.Destination}
(\emph{Nora a apporté le livre au metting}) avec l'objet direct avant le
syntagme prépositionnel. C'est à l'utilisateur de la ressource de considérer
que l'autre option (\emph{NP V PP.Destination}, \emph{Nora a apporté au meeting
le livre} est tout aussi valide.

\subsubsection{Travail au cas par cas}\label{subsubsec:casebycase}

Dans certains cas, cette deuxième étape pose des difficultés, et ce pour deux
raisons principales. Premièrement, certaines différences sémantiques entre
verbes communes à l'anglais et au français sont prises en comptes par VerbNet
mais ni par LVF ni par le LG. Par exemple, dans les verbes de \emph{Sendying
and Carrying} (la super-classe 11), les verbes dans les classes
{\color{blue}bring-11.3}, {\color{blue}carry-11.4} et {\color{blue}drive-11.5}
décrivent un mouvement accompagné (non seulement le Theme mais aussi l'Agent
changent de location comme dans \emph{Pamela drove packages to NY}). Au
contraire, les autres classes ({\color{blue}send-11.1} et
{\color{blue}slide-11.2} décrivent un mouvement non accompagné (seul le Thème
se déplace comme dans \emph{Pamela sent packages to NY}). Dans les ressources
françaises, des classes existent pour des verbes avec un changement de location
pour un Thème causé par un Agent, mais rien n'est codé pour le mouvement de
l'Agent. Face à cette difficulté, deux solutions se présentent.

\begin{itemize}
    \item soit réaliser une étude complète des verbes français de \emph{Sending
        and Carrying} pour distinguer les mouvements accompagnés et
        non-accompagnés,
    \item soit ignorer purement et simplement cette différence sémantique
\end{itemize}

Dans ce cas, nous avons opté pour la seconde solution étant donné que cette
information n'est pas directement utile pour l'annotation en rôles
sémantiques\footnote{Il semble par ailleurs que pour certains verbes anglais,
    l'Agent peut être ou non en mouvement : voir la différence entre les
classes VerbNet {\color{blue}carry-11.4} et {\color{blue}carry-11.4-1}.}.
C'est un choix discutable : si pour notre application la position de l'Agent
avait eu un intérêt, comme ce pourrait être le cas en implication textuelle,
une étude plus complète aurait été souhaitable. Nous préférons nous concentrer
sur une première version de VerbNet, quitte à revenir sur certains choix par la
suite.

Le fait d'ignorer cette différence nous mène à adopter dans \verbenet{} une
hiérarchie différente de VerbNet pour la super-classe 11 : il n'y a pas
d'équivalent dans \verbenet{} de la classe {\color{blue}carry-11.4}, les verbes
de cette classe devant être placés dans les classes ({\color{blue}send-11.1} et
{\color{blue}slide-11.2}. Par ailleurs, il n'y a pas d'équivalent en français
de la classe {\color{blue}bring-11.3} qui contient uniquement les deux verbes
\emph{bring} et \emph{take} avec une direction spécifiée déictiquement
\citep[page 135]{levin1993english} parce que les déictifs locatifs français
\emph{ici} et \emph{là} n'ont pas le fonctionnement de \emph{here} et
\emph{there} en anglais\footnote{\emph{Je suis là} peut signifier \emph{Je suis
ici}.}.

La seconde source principale de difficultés provient de différences cruciales
entre le français et l'anglais. Il existe des problèmes de traductions entre
ces deux langues qui sont bien connus et documentés, comme la traduction des
verbes de mouvement (par exemple \emph{John swam across the river}
$\rightarrow$ \emph{Jean a traversé la rivière à la nage}). Sans traiter ces
cas connus, nous discutons ici de situations plus subtiles, comme par exemple
avec les verbes de changement de possession. Dans VerbNet, dix classes sont
dédiées à ces verbes. Une telle hiérarchie est impossible en français. Sans
tout détailler, insistons sur ces quelques points :

\begin{itemize}

    \item L'absence d'alternances datif et bénéfactif en français implique que
        les classes VerbNet {\color{blue}give-13.1} et
        {\color{blue}contribute-13.2} doivent probablement être fusionnées en
        français.

    \item La différence sémantique entre {\color{blue}give-13.1} et
        {\color{blue}future\_having-13.3} (HAS-POSSESSION vs. FUTURE-POSESSION)
        est peut-être trop subtile et pourrait être ignorée.

    \item La préposition \emph{with} dans la frame correspondant à \emph{Agent
        V Recipient \{with\} Theme} utilisée en {\color{blue}fulfilling-13.4-1}
        et {\color{blue}fulfilling-13.4-2} doit être remplacée par \emph{en}
        et/ou \emph{de} suivant le verbe (e.g. \emph{Luc livre Max en/*de
        lait}, \emph{Luc équipe Max en/de téléviseurs}, \emph{Luc dote Max
        *en/de téléviseurs}, ce qui nécessite une réorganisation en
        sous-classes pour distinguer ces verbes.

\end{itemize}

En conclusion, il s'avère qu'en rentrant dans le détail des frames lors de
cette deuxième étape nous a mené a faire évoluer la hiérarchie de \verbenet{}.
Cependant, nous essayons de limiter les modifications au cas où il est
impossible de faire autrement afin de profiter au maximum de VerbNet et pour
pouvoir profiter du lien entre les deux ressources.

\subsection{Troisième étape}
\label{third}

La troisième étape, qui n'a pas encore commencé, est de valider manuellement
pour chaque classes les verbes proposés par correspondance de ressources en
supprimant les verbes erronés et en rajoutant les verbes manquant à fin que la
ressource ait été entièrement validée manuellement.

\section{Conclusion}

Nous avons présenté une méthode pour adapter la ressource syntaxico-sémantique
VerbNet vers une nouvelle langue. Cette méthode combine l'automatisation du
transfert de frames, la traduction automatique du lexique et une expertise
linguistique. Nous avons appliqué cette méthode au français et avons atteint un
point où cette ressource est validée et le travail systématique sur chaque
classe est en course. Nous reconnaissons les différences qui existent entre
les langues : la structure de \verbenet{} n'est pas exactement celle de
VerbNet.

% TODO libérer !
