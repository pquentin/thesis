\chapter{Traduction de VerbNet pour le français}
\label{ch:verbnet}


VerbNet est une ressource lexicale pour les verbes anglais organisée autour de classes sémantiques et de sous-classes syntaxiques : une classe sémantique est divisée en sous-classes de verbes qui partagent tous le même ensemble de cadres de sous-catégorisation et d'alternances, en suivant \cite{levin1993english}. Dans sa version actuelle (http://verbs.colorado.edu/~mpalmer/projects/verbnet.html), VerbNet comprend 3769 lemmes, donnant lieu à 5257 entrées réparties en 274 sous-classes. VerbNet a montré la cohérence de sa classification et est très utilisé, notamment pour l'annotation en rôles sémantiques (Swier et Stevenson, 2005 ; Palmer et al., 2013) où il présente l'intérêt de ne pas être restreint à un domaine spécifique tout en couvrant 99~\% des occurrences des verbes anglais dans un texte donné.

Il paraît donc nécessaire d'avoir une ressource équivalente pour le français. Les seuls efforts qui ont été faits dans cette direction pour l'instant se limitent à des constructions automatiques bruitées dont l'évaluation se limite à quelques verbes (Messiant et al., 2010 ; Falk et al., 2012). De plus ces efforts font abstraction des ressources lexicales qui existent pour le français, or celles–ci existent et sont de qualité. Pour les verbes, nous pensons en particulier à au Lexique des Verbes du français (LVF+1 \url{http://pageperso.lif.univ-mrs.fr/~paul.sabatier/Contribution_FondamenTAL.html}), au Lexique-Grammaire (LG http://infolingu.univ-mlv.fr/DonneesLinguisti ques/Lexiques-Grammaires) et à Dicovalence (http://bach.arts.kuleuven.be/dicovalence/). Nous avons donc l'objectif de réaliser un VerbeNet du français semi-automatiquement en nous appuyant sur ces ressources, en particulier sur LVF+1 et LG, la première plus centrée sur les informations sémantiques, la seconde sur les informations syntaxiques. Ce VerbeNet garde la hiérarchie des classes sémantiques du VerbNet anglais, ce qui permet de garder à l'identique les informations sémantiques, entre autres les rôles thématiques. Sa création demande d'accomplir deux tâches: pour chaque classe sémantique, déterminer ses membres puis les répartir en sous-classes syntaxiquement homogènes.

Pour la première tâche, la méthode est la suivante pour chaque classe de VerbNet notée Ce :
\begin{enumerate}
    \item nous associons manuellement la classe du LVF Clvf et la table du LG Clg correspondant à la définition sémantique de la classe Ce, (e.g. put-9.1 L3b 38LD).
    \item deux dictionnaires fournissent la liste Ltrad des traductions françaises des verbes peuplant la classe anglaise Ce, 
    \item les verbes de la classe française Cf sont a priori les verbes simples de Ltrad qui appartiennent à l'intersection de Clvf et Clg (e.g. mettre, poser ou installer pour put-9.1).
\end{enumerate}

Une interface permet de valider les résultats et d'ajouter à Cf des éléments de Ltrad qui n'appartiennent qu'à Clg et pas à Clvf par exemple. Pour la seconde tâche, nous devons procéder manuellement en nous servant principalement des informations syntaxiques codées dans le LG.

Cette création semi-automatique de VerbeNet est en cours. La première étape est terminée mais pas encore évaluée, elle a produit une ressource où les 247 classes sémantiques de VerbNet ont étés traduites, avec en moyenne 9 verbes français proposés par classe.

