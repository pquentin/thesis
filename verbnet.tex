\chapter{Traduction de VerbNet pour le français}
\label{ch:verbnet}

VerbNet est une ressource lexicale pour les verbes anglais organisée autour de classes sémantiques et de sous-classes syntaxiques : une classe sémantique est divisée en sous-classes de verbes qui partagent tous le même ensemble de cadres de sous-catégorisation et d'alternances, en suivant \cite{levin1993english}. Dans sa version actuelle\url{http://verbs.colorado.edu/~mpalmer/projects/verbnet.html}, VerbNet comprend 3769 lemmes, donnant lieu à 5257 entrées réparties en 274 sous-classes. VerbNet a montré la cohérence de sa classification et est très utilisé, notamment pour l'annotation en rôles sémantiques \citep{swier2005exploiting, palmer2013semantic} où il présente l'intérêt de ne pas être restreint à un domaine spécifique tout en couvrant 99~\% des occurrences des verbes anglais dans un texte donné.

Il paraît donc nécessaire d'avoir une ressource équivalente pour le français. Les seuls efforts qui ont été faits dans cette direction pour l'instant se limitent à des constructions automatiques bruitées dont l'évaluation se limite à quelques verbes \citep{messiant2010acquisition,falk2012classifying}. De plus ces efforts font abstraction des ressources lexicales qui existent pour le français, or celles–ci existent et sont de qualité. Pour les verbes, nous pensons en particulier à au Lexique des Verbes du français (LVF+1 \url{http://pageperso.lif.univ-mrs.fr/~paul.sabatier/Contribution_FondamenTAL.html}), au Lexique-Grammaire (LG \url{http://infolingu.univ-mlv.fr/DonneesLinguistiques/Lexiques-Grammaires}) et à Dicovalence (\url{http://bach.arts.kuleuven.be/dicovalence/}). Nous avons donc l'objectif de réaliser un VerbeNet du français semi-automatiquement en nous appuyant sur ces ressources, en particulier sur LVF+1 et LG, la première plus centrée sur les informations sémantiques, la seconde sur les informations syntaxiques. Ce VerbeNet garde la hiérarchie des classes sémantiques du VerbNet anglais, ce qui permet de garder à l'identique les informations sémantiques, entre autres les rôles thématiques. Sa création demande d'accomplir deux tâches: pour chaque classe sémantique, déterminer ses membres puis les répartir en sous-classes syntaxiquement homogènes.

Pour la première tâche, la méthode est la suivante pour chaque classe de VerbNet notée $Ce$ :
\begin{enumerate}
    \item nous associons manuellement la classe du LVF $C_{lvf}$ et la table du LG $C_{lg}$ correspondant à la définition sémantique de la classe $Ce$, (e.g. put-9.1 L3b 38LD).
    \item deux dictionnaires fournissent la liste $L_{trad}$ des traductions françaises des verbes peuplant la classe anglaise $Ce$, 
    \item les verbes de la classe française Cf sont a priori les verbes simples de $L_{trad}$ qui appartiennent à l'intersection de $C_{lvf}$ et $C_{lg}$ (e.g. mettre, poser ou installer pour put-9.1).
\end{enumerate}

Une interface permet de valider les résultats et d'ajouter à Cf des éléments de $L_{trad}$ qui n'appartiennent qu'à $_{Clg}$ et pas à $C_{lvf}$ par exemple. Pour la seconde tâche, nous devons procéder manuellement en nous servant principalement des informations syntaxiques codées dans le LG.

Cette création semi-automatique de VerbeNet est en cours. La première étape est terminée mais pas encore évaluée, elle a produit une ressource où les 247 classes sémantiques de VerbNet ont étés traduites, avec en moyenne 9 verbes français proposés par classe.

\section{VerbNet, une ressource syntaxico-sémantique}

\subsection{Les classes de Levin}

\cite{levin1993english} est une classification des verbes anglais suivant un principe simple : le comportement syntaxique des verbes détermine en partie leur sens. Après avoir défini un certain nombre de constructions syntaxiques possibles, les verbes sont classés en groupes partageant les mêmes constructions.

TODO exemple venant vraiment de Levin.

Cette classification a trois intérêts pour l'annotation en rôles sémantiques :
\begin{itemize}
    \item Une grande majorité des verbes anglais sont couverts, rendant la ressource utile pour des annotations à large échelle.
    \item La classification est hiérarchique et regroupe de nombreux verbes : avec une cinquantaine de classes principales, des généralisations et partages d'informations entre les classes proches sont possibles.
    \item Les comportements syntaxiques déterminent les comportements sémantiques, ce qui correspond au schéma classique de l'annotation en rôles sémantiques qui s'appuie sur une analyse syntaxique.

\subsection{Intersection des classes de Levin}

TODO estimer la pertinence

Le comportement syntaxique des verbes anglais détermine leur sens, mais pas entièrement. En effet, les classes de Levin ne s'intéressent pas directement au sens, et regroupent ainsi des verbes donnant l'impression d'être déconnectés. Ainsi, dans les \cite{put verbs} se trouve la classe \cite{Pocket} qui regroupe les verbes "mettre dans sa poche" et "mettre en prison".

TODO exemple et intérêt pour nous : tirer des exemples de plusieurs classes ? Restreindre les frames possibles ?

\subsection{VerbNet}

\subsection{VerbNet et universaux linguistiques}

Portuguais, français, ...
