% vim: set spelllang=fr:
\chapter{\verbenet{} : une traduction de VerbNet}
\label{ch:verbnet}

% L'introduction doit tenir dans un petit paragraphe qui fait référence à
% d'autres sections
La ressource VerbNet (section~\ref{subsec:presentation_verbnet}) correspond à
nos objectifs (section~\ref{objectifs_these}) mais n'existe qu'en anglais :
c'est pourquoi nous présentons dans ce chapitre sa traduction vers le français,
\verbenet{}\footnote{Le nom vient de la prononciation à la française qui
"rajoute" un \emph{e}. Le $\ni$ est là pour bien marquer ce \emph{e} et
ainsi essayer d'éviter la confusion avec VerbNet.}.

Nous avons d'abord (section~\ref{first}) traduit vers le français les membres
des classes de premier niveau dans VerbNet en utilisant des ressources
linguistiques qui encodent des informations syntaxiques et sémantiques sur les
verbes du français. La deuxième étape (section~\ref{second}), toujours en
cours, est d'adapter les cadres de sous-catégorisation vers le français et de
réorganiser les classes de VerbNet en fonction, ce qui pose divers problèmes
que nous décrirons en partie ici. La troisième étape (section~\ref{third}) sera
l'occasion de valider les membres français de chaque classe.

Notre traduction de VerbNet sera liée aux deux ressources linguistiques que
nous utilisons, Les Verbes Français et le Lexique-Grammaire. La ressources sera
aussi ouverte : nous voulons encourager les cosntributions externes avec notre
outil basé sur le web. Nous souhaitons aussi faciliter l'utilisation de la
ressource en utilisant le même format XML que le VerbNet anglais.

\section{Construction de \verbenet{}}

% TODO zero-related to a noun = dénominal ?
Notre principe de base est que la hiérarchie supérieure de VerbNet doit être
aussi proche que possible de celle de VerbNet et ses 270 classes. Néanmoins,
certaines classes peuvent disparaître. Ceci peut être dû à des raisons purement
morphologiques. Une classe VerbNet ne contenant que des verbes dénominaux n'a
pas d'équivalent en français. C'est le cas de \texttt{pit-10.7} avec des verbes
tels que \emph{bark} et \emph{bone} ou \texttt{week-end-56} et ses verbes
\emph{week-end} ou \emph{december}. Par contre, \texttt{debone-10.8} et ses
verbes formés par le préfixe \emph{dé-} plus un nominal (\emph{debark, debone})
a un équivalent français avec les préfixes \emph{dé-} ou \emph{é}
(\emph{déveiner}, \emph{équeuter}).

Étant donné ce principe de bases, la construction de \verbenet{} se fait en
trois étapes.

\subsection{Première étape}\label{first}

La première étape pour construire \verbenet{} est de déterminer quels verbes
français appartiennent à une des 270 classes de VerbNet. Ceci a été fait en
trois étapes.

\begin{enumerate}

    \item Pour une classe VerbNet donnée \Ce{}, nous assignons manuellement les
        classes LVF \Clvf{} et les classes LG \Clg{} qui correspondent à sa
        définition sémantique : nous assignons par exemple {\color{red}L3b} et
        {\color{green}38LD} à {\color{blue}put-9.1} et assignons
        {\color{red}M1a} et {\color{green}32CL ou 32R3 ou 32C}) à
        {\color{blue}body\_internal\_motion-49}.

    \item Nous utilisons deux dictionnaires bilingues (SCI-FRAND-EURADIC et le
        Wiktionnaire) qui fournissent la liste \Ltrad{} des traductions
        françaises des verbes anglais appartenant à \Ce{} et ses sous-classes.

    \item Les verbes français de la classe \Ce{} sont alors simplement les
        verbes appartenant à la fois à \Ltrad{}, \Clvf{} et
        \Clg{}\footnote{Quand cette intersection est vide, c'est la liste
        non-vide (soit \Clvf{} soit \Clg{}) qui est choisie}. Par exemple,
        \emph{mettre}, \emph{poser} et \emph{installer} sont des verbes
        français de {\color{blue}put-9.1}).

\end{enumerate}

Cette étape a été exécutée rapidement et a donné des résultats encourageants :
en ne gardant que les verbes à l'intersection de \Ltrad{}, \Clvf{} et \Clg{},
les résulstats sont précis, et cohérents d'un point de vue sémantique et
syntaxique. Par exemple, la classe {\color{blue}scribble-25.2} contient 18
verbes en anglais : elle est associée à la classe LVF {\color{red}R3a.1} et à
la table LG {\color{green}32A}, ce qui conduit à une liste de 16 verbes
français : \emph{composer}, \emph{couper}, \emph{donner}, \emph{exécuter},
\emph{fabriquer}, \emph{faire}, \emph{forger}, \emph{former}, \emph{imprimer},
\emph{lever}, \emph{produire}, \emph{reproduire}, \emph{sculpter},
\emph{tailler}, \emph{tirer} et \emph{tracer}. Tous ces verbes sont valides
pour cette classe.  Cette méthode produit un lexique de 4058 verbes (2128
verbes distincts).

% TODO An evaluation on 10 randomly chosen classes shows that XX.X\% of verbs are indeed valid.


\subsection{Deuxième étape}\label{second}

% TODO stats sur un VerbNet importé, pas sur VerbeNet à moitié fini
La deuxième étape de la construction de \verbenet{} s'est avérée beaucoup plus
fastidieuse que la première. Pour chacune des 500 classes et sous-classes, nous
déterminons quand cela est possible :

\begin{itemize}

    \item les frames valides pour le français avec des ajustements possibles
        pour les rôles thématiques et les restrictions de sélection,

    \item les sous-classes en français en réorganisant si besoin la hiérarchie
        de l'anglais dans le but d'assigner les verbes obtenus à l'étape 1 à
        une des sous-classes.

\end{itemize}

Cette étape a demandé le développement d'un outil d'édition
(section~\ref{toolquentin}) pour assister le travail lexicographique. Ensuite,
il a fallu définir des principes de base sur les frames françaises quand elle
diffèrent des frames anglaises (section~\ref{princp}). Finalement, une étude
fine au cas par cas révèle des différences difficiles à traiter entre l'anglais
et le français (section~\ref{subsubsec:casebycase}).

\subsubsection{Outil d'édition de \verbenet{}}\label{toolquentin}

Nous avons du développer un outil permettant d'éditer collaborativement
VerbNet. Cet outil est en fait un site web présentant VerbNet : ses classes,
ses frames, ses verbes, etc. Tous les éléments sont modifiables, ce qui permet
aux lexicographes de se concentrer sur les problèmes lexicographiques sans
avoir à se préoccuper de la représentation des données. Le site web est réalisé
avec le framework web Python Django. VerbNet est stocké dans une base de donnée
PostgreSQL. Avant de commencer l'édition, nous avons commencé par charger
VerbNet et les mappings réalisés lors de la première étape. L'édition elle-même
se fait par classe de Levin. Par exemple, nous avons traité les classes
{\color{blue}throw-17.1} et {\color{blue}pelt-17.2} parce qu'elles font toutes
les deux partie de la «~super-classe~» 17 : Throwing. Plusieurs classes faisant
partie d'une même super-classe sont souvent liées et il faut les comprendre
dans leur ensemble avant de commencer l'édition. Pour chaque classes, nous
pouvons éditer ses frames, ajouter ou supprimer des sous-classes ou des frames.
Par exemple, toutes les frames impliquant un conatif, un datif ou une
alternance bénéfactive (?) peuvent être systématiquement supprimées : ces
alternances n'existent pas en français.

% TODO faire une nouvelle capture
\begin{figure}
    \centering
    \includegraphics[width=0.48\textwidth]{fig/tool_screenshot.png}
    \caption{\label{tool}Interface web pour analyser et modifier \verbenet{}.
    Chaque frame peut être modifiée et les classes peuvent être réorganisées. Les
    traductions en violet apaprtiennent à l'intersection de \Clvf{}, \Clg{} et
    \Ltrad{} (section~\ref{first}), les traductions en rouge (respectivement en
    vert) uniquement à \Clvf{} et \Ltrad (respectivement uniquement à \Clg{} et
    \Ltrad{}).}
\end{figure}

Avec l'aide de cet outil (illustré à la Figure~\ref{tool}), la deuxième étape
peut s'avérer très simple. Par exemple, les quatre sous-classes de
{\color{blue}imaeg-creation-25} ont des classes directement équivalentes en
français, donc les seules choses à faire sont de traduire les exemples en
français avec les bonnes prépositions. Par exemple, dans la classe
{\color{blue}illustrate-25.3}, il a fallu remplacer \emph{with} par la
combinaison de \emph{de} et \emph{avec}.

\subsubsection{Principes sur les frames}\label{princp}

Nous avons jusqu'ici identifié deux différences principales entre le codage des
frames anglaises et françaises.

La première différence concerne les sous-structures, c'est-à-dire les frames
avec un complément manquant tel que \emph{NP V} (Luc gravait) dans
{\color{blue}image-impression-25.1}. C'est en effet ici une sous-structure de
\emph{NP V NP.Destination} (Luc gravait les annaux). Le codage de telles
sous-structures est difficile à justifier quand il est basé sur l'introspection
linguistique et nécessite une étude de corpus. Nous ne savons pas comment ce
codage a été fait dans VerbNet et n'avons pas à notre disposition de corpus
français permettant de répondre à la question. Nous avons donc décidé pour le
moment de supprimer toutes les sous-structures de \verbenet{}. Par exemple,
dans la classe {\color{blue}remove-10.1}, VerbNet encode non seulement NP V NP
PP.Source PP.Destination (\emph{Doug removed the smudges from the tabletop})
mais aussi NP V NP (\emph{Doug removed the smudges}). \verbenet{} n'inclut que
la première frame, il est implicite que la seconde existe : une application
doit donc l'inférer automatiquement à partir de la première sans intervention
manuelle\footnote{Cependant, ce principe ne s'applique pas aux verbes acceptant
un seul complément locatif double «~from here to there (un seul
complément PP.Source PP.Destination)~» sans accepter un seul complément source
(PP.Source), tout en acceptant un seul complément destination
(PP.Destination) : \emph{Fred a transferré le vin de la cruche en pierre vers
la cruche en terre cuite, *Fred a transferré le vin de la cruche en pierre,
Fred a transferré le vin vers la cruche en terre cuite}. Dans ces cas, les
sous-structures sont codées}.

La deuxième différence concerne l'ordre des compléments. VerbNet encode parfois
deux frames qui ne diffèrent que par l'ordre des compléments, par exemple dans
{\color{blue}bring-11.3} les frames NP V NP PP.Destination NP (\emph{Nora
brought to lunch the book}) et NP V PP.Destination NP (\emph{Nora brought to
lunch the book}). En français, l'ordre des compléments dépend d'un certain
nombre de facteurs syntaxiques et sémantiques \citep{thuilier2012contraintes},
mais ne dépend pas a priori d'un facteur lexical : il ne dépend pas du verbe
qui gouverne les compléments. C'est pour cette raison que \verbenet{} n'encode
qu'un frame dans ces cas, ici seulement \emph{NP V NP PP.Destination}
(\emph{Nora a apporté le livre au metting}) avec l'objet direct avant le
syntagme prépositionnel. C'est à l'utilisateur de la ressource de considérer
que l'autre option (\emph{NP V PP.Destination}, \emph{Nora a apporté au meeting
le livre} est tout aussi valide.

\subsubsection{Travail au cas par cas}\label{subsubsec:casebycase}

Dans certains cas, cette deuxième étape pose des difficultés, et ce pour deux
raisons principales. Premièrement, certaines différences sémantiques entre
verbes communes à l'anglais et au français sont prises en comptes par VerbNet
mais ni par LVF ni par le LG. Par exemple, dans les verbes de \emph{Sendying
and Carrying} (la super-classe 11), les verbes dans les classes
{\color{blue}bring-11.3}, {\color{blue}carry-11.4} et {\color{blue}drive-11.5}
décrivent un mouvement accompagné (non seulement le Theme mais aussi l'Agent
changent de location comme dans \emph{Pamela drove packages to NY}). Au
contraire, les autres classes ({\color{blue}send-11.1} et
{\color{blue}slide-11.2} décrivent un mouvement non accompagné (seul le Thème
se déplace comme dans \emph{Pamela sent packages to NY}). Dans les ressources
françaises, des classes existent pour des verbes avec un changement de location
pour un Thème causé par un Agent, mais rien n'est codé pour le mouvement de
l'Agent. Face à cette difficulté, deux solutions se présentent.

\begin{itemize}
    \item soit réaliser une étude complète des verbes français de \emph{Sending
        and Carrying} pour distinguer les mouvements accompagnés et
        non-accompagnés,
    \item soit ignorer purement et simplement cette différence sémantique
\end{itemize}

Dans ce cas, nous avons opté pour la seconde solution étant donné que cette
information n'est pas directement utile pour l'annotation en rôles
sémantiques\footnote{Il semble par ailleurs que pour certains verbes anglais,
    l'Agent peut être ou non en mouvement : voir la différence entre les
classes VerbNet {\color{blue}carry-11.4} et {\color{blue}carry-11.4-1}.}.
C'est un choix discutable : si pour notre application la position de l'Agent
avait eu un intérêt, comme ce pourrait être le cas en implication textuelle,
une étude plus complète aurait été souhaitable. Nous préférons nous concentrer
sur une première version de VerbNet, quitte à revenir sur certains choix par la
suite.

Le fait d'ignorer cette différence nous mène à adopter dans \verbenet{} une
hiérarchie différente de VerbNet pour la super-classe 11 : il n'y a pas
d'équivalent dans \verbenet{} de la classe {\color{blue}carry-11.4}, les verbes
de cette classe devant être placés dans les classes ({\color{blue}send-11.1} et
{\color{blue}slide-11.2}. Par ailleurs, il n'y a pas d'équivalent en français
de la classe {\color{blue}bring-11.3} qui contient uniquement les deux verbes
\emph{bring} et \emph{take} avec une direction spécifiée déictiquement
\citep[page 135]{levin1993english} parce que les déictifs locatifs français
\emph{ici} et \emph{là} n'ont pas le fonctionnement de \emph{here} et
\emph{there} en anglais\footnote{\emph{Je suis là} peut signifier \emph{Je suis
ici}.}.

La seconde source principale de difficultés provient de différences cruciales
entre le français et l'anglais. Il existe des problèmes de traductions entre
ces deux langues qui sont bien connus et documentés, comme la traduction des
verbes de mouvement (par exemple \emph{John swam across the river}
$\rightarrow$ \emph{Jean a traversé la rivière à la nage}). Sans traiter ces
cas connus, nous discutons ici de situations plus subtiles, comme par exemple
avec les verbes de changement de possession. Dans VerbNet, dix classes sont
dédiées à ces verbes. Une telle hiérarchie est impossible en français. Sans
tout détailler, insistons sur ces quelques points :

\begin{itemize}

    \item L'absence d'alternances datif et bénéfactif en français implique que
        les classes VerbNet {\color{blue}give-13.1} et
        {\color{blue}contribute-13.2} doivent probablement être fusionnées en
        français.

    \item La différence sémantique entre {\color{blue}give-13.1} et
        {\color{blue}future\_having-13.3} (HAS-POSSESSION vs. FUTURE-POSESSION)
        est peut-être trop subtile et pourrait être ignorée.

    \item La préposition \emph{with} dans la frame correspondant à \emph{Agent
        V Recipient \{with\} Theme} utilisée en {\color{blue}fulfilling-13.4-1}
        et {\color{blue}fulfilling-13.4-2} doit être remplacée par \emph{en}
        et/ou \emph{de} suivant le verbe (e.g. \emph{Luc livre Max en/*de
        lait}, \emph{Luc équipe Max en/de téléviseurs}, \emph{Luc dote Max
        *en/de téléviseurs}, ce qui nécessite une réorganisation en
        sous-classes pour distinguer ces verbes.

\end{itemize}

En conclusion, il s'avère qu'en rentrant dans le détail des frames lors de
cette deuxième étape nous a mené a faire évoluer la hiérarchie de \verbenet{}.
Cependant, nous essayons de limiter les modifications au cas où il est
impossible de faire autrement afin de profiter au maximum de VerbNet et pour
pouvoir profiter du lien entre les deux ressources.

\subsection{Troisième étape}
\label{third}

La troisième étape, qui n'a pas encore commencé, est de valider manuellement
pour chaque classes les verbes proposés par correspondance de ressources en
supprimant les verbes erronés et en rajoutant les verbes manquant à fin que la
ressource ait été entièrement validée manuellement.

\section{Conclusion}

Nous avons présenté une méthode pour adapter la ressource syntaxico-sémantique
VerbNet vers une nouvelle langue. Cette méthode combine l'automatisation du
transfert de frames, la traduction automatique du lexique et une expertise
linguistique. Nous avons appliqué cette méthode au français et avons atteint un
point où cette ressource est validée et le travail systématique sur chaque
classe est en course. Nous reconnaissons les différences qui existent entre
les langues : la structure de \verbenet{} n'est pas exactement celle de
VerbNet.

% TODO libérer !
