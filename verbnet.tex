% vim: set spelllang=fr:
\chapter{Traduction de VerbNet pour le français}
\label{ch:verbnet}

\section{Introduction}

% TAL a besoin d'infos sur les verbes, c'est difficile et fragile, et on a
% besoin d'énormes corpus
Le Traitement Automatique des Langues requiert des lexiques et de larges
quantités de données annotées pour analyser efficacement des textes dans le
domaine général. Obtenir cette quantité de données est un problème en soi connu
sous le nom de "knowledge acquisition bottleneck" en désambiguïsation lexicale
\citep{gale1992using}. Annoter de plus en plus de données de manière efficace
réduira le bottleneck pour certains domaines, mais d'autres stratégies sont
nécessaires pour atteindre nos objectifs. Une possibilité est d'utiliser au
mieux les données annotées en perfectionnant les algorithmes existants, une
autre est d'utiliser intelligemment les données non annotées qui existent en
quantité bien plus importante. Une troisième possibilité selon nous est
d'encoder les lexiques de manière efficiente pour couvrir une large partie du
vocabulaire. C'est ce qui est fait dans VerbNet où les traits partagés par les
mêmes verbes sont explicitement notés, ce qui permet à chaque modification dans
VerbNet de mieux traiter un certain nombre de verbes au lieu d'un seul.

% VerbNet a montré son efficacité dans le domaine de par son approche
% pragmatique. Vient des classes de Levin, donne des correspondances entre
% syntaxe et sémantique
Deux difficultés majeures qu'affrontent les créateurs de lexique sont la
granularité de sens et la distinction des sens. Ces deux difficultés sont
gérées par les classes de Levin \citep{levin1993english}. Dans ces classes, les
verbes sont classifiés principalement à travers leur alternances syntaxiques,
ce qui fournit un critère qui est à la fois facilement observable et qui
produit des distinctions sémantiques intéressantes. VerbNet
\citep{kipperschuler2005verbnet} est un lexique électronique basé sur les
classes de Levin. Il encode non plus les alternances mais les cadres de
sous-catégorisation valables pour chaque classe, et rajoute des informations de
rôle et de sémantique à travers une logique des prédicats simplifiée. De
nouvelles classes, constructions et verbes ont étés ajoutés à VerbNet au fil
des ans.

% VN est bon pour SRL
VerbNet est un lexique adapté au Traitement Automatique des Langues : on peut
utiliser un cadre de sous-catégorisation pour mapper des syntagmes à des rôles
thématiques \citep{swier2005exploiting,pradet2013revisiting}. Grâce à sa
couverture élevées (plus de quatre mille verbes distincts) et son groupement de
verbes utile, VerbNet est bien adapté à l'annotation en rôles sémantiques.

% Dans les langues autres que l'anglais, cette ressource utile n'existe pas
% mais ne demande qu'à exister étant donné le potentiel \textit{cross-lingual}
% de VerbNet. Il y a souvent des ressources proches, moins utiles, plus
% linguistiques, existent.
Cependant, un VerbNet de qualité n'existe pour le moment que pour l'anglais.
Une telle ressource serait pourtant encore plus utile pour les langues moins
ressourcées où les corpus annotés en rôles sémantiques n'existent pas. VerbNet
a un potentiel inter-linguistique, visible notamment avec le portuguais
\citep[section 2.2.2]{kipperschuler2005verbnet}. Adapter VerbNet vers une
nouvelle langue suffisamment proche de l'anglais permet de conserver sa
structure, ainsi que l'information sémantique et les rôles thématiques, ce qui
donne la possibilité de produire un lexique utile sans des années de travail
manuel.

% En particulier, en français, les tables du LADL décrivent le comportement
% syntaxique des verbes et le LVF découpe les verbes du français
% sémantiquement. Un certain nombre de difficultés empêche d'utiliser ces
% ressources en TAL

En ayant pour but de développer une version français de VerbNet (nommée
\verbenet{}), nous avons d'abord (section~\ref{first}) traduit vers le français
les membres des classes de premier niveau dans VerbNet en utilisant des
ressources linguistiques qui encodent des informations syntaxiques et
sémantiques sur les verbes du français. La deuxième étape
(section~\ref{second}), toujours en cours, est d'adapter les cadres de
sous-catégorisation vers le français et de réorganiser les classes de VerbNet
en fonction, ce qui pose divers problèmes que nous décrirons en partie ici. La
troisième étape (section~\ref{third}) sera l'occasion de valider les membres
français de chaque classe.

Notre traduction de VerbNet sera liée aux deux ressources linguistiques que
nous utilisons, Les Verbes Français et le Lexique-Grammaire. La ressources sera
aussi ouverte : nous voulons encourager les cosntributions externes avec notre
outil basé sur le web. Nous souhaitons aussi faciliter l'utilisation de la
ressource en utilisant le même format XML que le VerbNet anglais.

\section{Présentation de VerbNet et des ressources lexicales françaises}
\label{french}

\subsection{Les classes de Levin}

% Bien distinguer alternances vs. cadres de sous-cat

\cite{levin1993english} est une classification des verbes anglais suivant un
principe simple : le comportement syntaxique des verbes détermine en partie
leur sens. Après avoir défini un certain nombre d'alternances de diathèses
possibles, les verbes sont classés en groupes partageant les mêmes
constructions.

TODO exemple d'alternance venant vraiment de Levin.

Cette classification a trois intérêts pour l'annotation en rôles sémantiques :

\begin{itemize}

    \item Une grande majorité des verbes anglais sont couverts, rendant la
        ressource utile pour des annotations à large échelle.

    \item La classification est hiérarchique et regroupe de nombreux verbes :
        avec une cinquantaine de classes principales, des généralisations et
        partages d'informations entre les verbes de classes identiques ou
        proches sont possibles.

    \item Les comportements syntaxiques déterminent les comportements
        sémantiques, ce qui correspond au schéma classique de l'annotation en
        rôles sémantiques qui s'appuie sur une analyse syntaxique.

\end{itemize}

%\subsection{Intersection des classes de Levin}

% TODO raccourcir sans dire de bêtises

Les classes de Levin ne s'intéressent pas directement au sens, et regroupent
parfois des verbes dont le sens peut être très différent.  Ainsi, dans les
\textit{put verbs} se trouve la classe \textit{Pocket} qui regroupe les verbes
"mettre dans sa poche" et "mettre en prison". Cependant, le comportement
syntaxique est le même : le regroupement est logique et les différences de sens
ne sont a priori pas gênantes pour nos applications qui ne se basent pas sur le
sens des verbes dans VerbNet.

Pour affiner le sens des classes obtenues, \cite{dang1998investigating} propose
les classes de Levin intersectives en donnant l'exemple d'intersections de
classes qui donnent des ensembles sémantiquement cohérents. Ainsi, les
\textit{split verbs} sont définis comme des verbes séparant par une action en
\textit{-ing}. Ces verbes incluent \textit{cut}, \textit{rip}, \textit{split}
mais aussi \textit{push}, \textit{kick} ou encore \textit{slip}, étant donné
que \textit{push} par exemple sépare de par l'action de pousser. Un autre
exemple : les \textit{carry verbs} sont définis comme les verbes causant un
mouvemant, avec une séparation entre les mouvements accompagnés
(\textit{carry}, \textit{drag}, ...) et le mouvement non accompagné
(\textit{shove}, \textit{drag}, ...). Au-delà de cette distinction, c'est une
classe qui contient deux types de verbes : ceux qui acceptent une
\textit{conative alternation} indiquant que le mouvement est essayé sans être
réussi et ceux qui impliquent seulement que le mouvement est causé par l'agent.

En combinant les verbes apparaissant à la fois dans les \textit{split verbs} et
les \textit{carry verbs}, on obtient un ensemble sémantiquement cohérent :
uniquement les verbes indiquant l'application de la force du côté de
\textit{pull}, et uniquement les verbes partageant la \textit{conative
alternation} pour \textit{carry}. Cette opération peut être automatisée : dès
que deux classes partagent au moins trois verbes, \cite{dang1998investigating}
a jugé intéressant de créer une nouvelle classe contenant l'intersection qui
sera cohérente sémantiquement et collera mieux avec la ressource sémantique
WordNet.

% TODO exemple et intérêt pour nous : tirer des exemples de plusieurs classes ?
% Restreindre les frames possibles ?

\subsection{VerbNet}

VerbNet est une extension modernes des classes de Levin, qui se retrouvent
améliorées sur plusieurs fronts :

\begin{itemize}

    % TODO vraiment ?
    \item la couverture des verbes existants a été testée sur PropBank, ce qui
        a mené à l'ajout de nouvelles constructions syntaxiques

    \item de nouvelles classes provenant de \cite{korhonen2004extended}
        intégrant les verbes acceptant des complétives, mais aussi des
        syntagmes adjectivaux et adverbiaux ou encore des particules.

    \item de nouveaux verbes provenant de \cite{dorr2001lcs} ont étés intégrés

    \item les verbes ont étés liés à WordNet, OntoNotes, PropBank et FrameNet
        \citep{palmer2009semlink}

    \item de nombreuses corrections ont étés apportées au fil des versions.

\end{itemize}

Ces améliorations ont à la fois contribué à VerbNet en largeur (nouvelles
classes) et en profondeur (nouveaux verbes, nouvelles constructions
syntaxiques).

% TODO nombre de sous-classes
% TODO nombre de verbes
La hiérarchie supérieure de VerbNet contient 270 classes. Certaines de ces
classes sont sous-divisées pour former une hiérarchie relativement plate :
VerbNet contient en tout XXX classes. Pour chaque (sous-)classe, ce lexique
indique :

\begin{itemize}
        \item la liste des verbes de la classe,
        \item les rôles thématiques en jeu ainsi que leur restrictions de sélection,
        \item la liste des \emph{frames} VerbNet.
\end{itemize}

Une frame inclut une phrase d'exemple, une formule syntaxique donnant la
liaison entre les syntagmes et les rôles thématiques, une formule sémantique
basée sur la logique des prédicats explicitant la relation entre les
participants et les évènements.

% TODO ?

\subsection{Les Verbes Français et Lexique-Grammaire}

Un des intérêts des classes de Levin et de VerbNet par rapport aux ressources
françaises est leur approche pragmatique qui se traduit notamment par l'absence
de prise en compte des emplois métaphorique d'un verbe donné. Ainsi, les tables
du LADL incluent des usages tels que \textit{L'idée gallopait dans son esprit}
qui peuvent induire en erreur une application. \cite{brown2012semantic}
proposent une analyse systématique des emplois métaphorique de deux verbes
représentatifs et montrent qu'utiliser VerbNet pour raisonner sur les emplois
métaphorique d'un texte est en partie possible au prix d'une complexité plus
importante et de prédicats sémantiques moins précis.

À partir des années 70 deux ressources lexicales pour les verbes français ont
été dévelopées : LVF et LG. \footnote{Plus tard, dans les années 1990, une
autre ressource a été dévelopée : Dicovalence. Nous ne l'utilisons presque pas
dans nos travaux.}

\begin{itemize}

    % TODO mieux que sous-sous-sous et nombre de E2f.1
    \item LVF (Les Verbes Français, \cite{dubois1997verbes}) contient environ
        25000 entrées classées en 14 classes sémantiques, 54 sous-classes
        syntactico-sémantiques, 248 sous-sous-classes et XXX
        sous-sous-sous-classes.

    \item LG (Lexique-Grammaire, \cite{gross1975methodes,boons1976structure})
        comporte lui 14 000 entrées classifiées en 67 «~tables~», chaque table
        groupant des verbes partageant la même propriété définitoire syntaxique
        et potentiellement une sémantique similaire. Chaque colonne de la table
        encode des restrictions supplémentaires s'appliquant à certains des
        verbes de la table.

    %\item DICOVALENCE \citep{} est basé sur l'approche pronominale \citep{} :
    %    ce lexique liste 5 000 entrées parmi les plus fréquentes en français.

\end{itemize}

Les classes LVF et les tables LG peuvent toutes les deux être comparées aux
classes VerbNet. Cependant, ces (vieilles) ressources n'encodent ni les rôles
thématiques ni les formules sémantiques \footnote{Les notions de rôles
thématiques et d'évènement n'étaient pas répandues dans les années 1970.}.
C'est la raison pour laquelle nous voulons construire une nouvelle ressource
française, \verbenet{}\footnote{Le nom vient de la prononciation à la française
qui "rajoute" un e. Le $\ni$ est simplement là pour bien marquer ce e et ainsi
essayer d'éviter la confusion avec VerbNet}. Cette ressource tire profit d'une
part des ressources existantes pour le français avec un encodage sémantique et
syntaxique riche, et d'autre part de l'information sémantique présente dans
VerbNet pour l'anglais, une langue proche du français.

\section{Construction de \verbenet{}}

% TODO zero-related to a noun = dénominal ?
Notre principe de base est que la hiérarchie supérieure de VerbNet doit être
aussi proche que possible de celle de VerbNet et ses 270 classes. Néanmoins,
certaines classes peuvent disparaître. Ceci peut être dû à des raisons purement
morphologiques. Une classe VerbNet ne contenant que des verbes dénominaux n'a
pas d'équivalent en français. C'est le cas de \texttt{pit-10.7} avec des verbes
tels que \emph{bark} et \emph{bone} ou \texttt{week-end-56} et ses verbes
\emph{week-end} ou \emph{december}. Par contre, \texttt{debone-10.8} et ses
verbes formés par le préfixe \emph{dé-} plus un nominal (\emph{debark, debone})
a un équivalent français avec les préfixes \emph{dé-} ou \emph{é}
(\emph{déveiner}, \emph{équeuter}).

Étant donné ce principe de bases, la construction de \verbenet{} se fait en
trois étapes.

\subsection{Première étape}\label{first}

The first step in building \verbenet{} was to determine which French verbs
belong to one of  VerbNet's 270 classes. This was done in three stages:

\begin{enumerate}

    \item For a given VerbNet class \Ce{}, we  manually assigned the LVF class(es)
    \Clvf{} and the LG's table(s) \Clg{} that fit its semantic definition (e.g.
    {\color{blue}put-9.1} {\color{red}L3b} {\color{green}38LD} or
    {\color{blue}body\_internal\_motion-49} {\color{red}M1a} {\color{green}32CL or
    32R3 or 32C}),

    \item we used two bilingual dictionaries (SCI-FRAN-EURADIC and the French
    Wiktionary) which give the list \Ltrad{} of the French translations of the
    English verbs belonging to \Ce{} or a subclass of \Ce{},

    \item we computed the verbs of the French class \Cf{} which are a priori the
    simple verbs of \Ltrad{} which belong to the intersection of \Clvf{} and \Clg{}
    (e.g. \emph{mettre, poser} or \emph{installer} in {\color{blue}put-9.1}).

\end{enumerate}

This step was performed quickly and gave accurate results: by keeping only
verbs at the intersection of \Ltrad{}, \Clvf{} and \Clg{}\footnote{When the
intersection is empty, the non-empty list (\Clvf{} or \Clg{}) was chosen.}, the
results are precise and syntactically and semantically coherent. For example,
the {\color{blue}scribble-25.2} class contains 18 verbs in English; it is
associated with LVF {\color{red}R3a.1} and LG {\color{green}32A},  which leads to a
list of 16 French verbs: \emph{composer}, \emph{couper}, \emph{donner},
\emph{exécuter}, \emph{fabriquer}, \emph{faire}, \emph{forger}, \emph{former},
\emph{imprimer}, \emph{lever}, \emph{produire}, \emph{reproduire},
\emph{sculpter}, \emph{tailler}, \emph{tirer} and \emph{tracer}. All these
verbs are valid for this class.  This method results in a lexicon with 4058
verbs (2128 distinct verbs).

% TODO An evaluation on 10 randomly chosen classes shows that XX.X\% of verbs are indeed valid.


\subsection{Deuxième étape}\label{second}

The second step in building \verbenet{} has proven much more difficult than the
first. For each of the 270 \Cf{} sub-classes, we determine whenever possible:

\begin{itemize}

    \item the possible subclasses in order to assign the verbs found in the
    first step to one of these sub-classes (if possible)

    \item the frames that are valid for French with possible adjustments for
    thematic roles and selection restrictions.

\end{itemize}

This step has first required to develop an editing tool
(Section~\ref{toolquentin}) to help and maintain the lexicographers' work.
Next, it has required to set up basic principles on French frames, when they
differ from English ones (Section~\ref{princp}). Finally, a fine grained
case-by-case study reveals some tough differences between French and English,
which are illustrated in (Section~\ref{subsubsec:casebycase}).


\subsubsection{\verbenet{} editing tool}\label{toolquentin}

This step  required us to develop a web-based tool which makes it possible to
collaboratively edit VerbNet classes and frames by manipulating their
representation on the website. This online interface developed with Django (a
Python web framework) hides a PostgreSQL database that stores all this
information and tracks all changes to the data.  The tool was first filled with
VerbNet frames and verb translations found in the first step. It allows
us to edit a frame and to suppress or add a (sub-)class or a frame. For
example, all the frames involving a conative, dative or benefactive alternation
can be systematically suppressed because these alternations don't exist in
French.

\begin{figure}
 \centering
 \includegraphics[width=0.48\textwidth]{fig/tool_screenshot.png}
 \caption{\label{tool}Web interface to analyze and edit \verbenet{}. Every
    frame can be modified and the structure can be reorganized.  The translations
    in purple belong to the intersection of \Clvf{} and \Clg{}; the translations in
    red (resp. green) belong only to \Clvf{} (resp. \Clg{}).}
\end{figure}


With the help of this tool (illustrated in Figure~\ref{tool}),  the work for
the second step can be very easy. For example, the four sub-classes of
image-creation-25 have direct equivalent classes in French, so the only thing
to do is to provide French examples with the right preposition(s), e.g.
\emph{with} in 25.3 has to be replaced in French with \emph{de} or \emph{avec}.

\subsubsection{Principles on frames}\label{princp}

So far, we have found two general differences between the  coding of French and
English frames in \verbenet{} and VerbNet respectively.

The first one concerns ``sub-frames'', i.e. frames with missing complements such
as \emph{NP V} in 25.1 illustrated by \emph{Smith was inscribing} which could
be  a sub-frame of e.g. \emph{NP V NP.destination} (\emph{Smith was inscribing
the rings}). The coding of such sub-frames is dubious when based on
introspection so it requires some corpus study.  We don't know how this coding
has been made in VerbNet and we don't have at our disposal enough French corpus
data.  So we decided for the time being to remove sub-frames from \verbenet{}.
For example in class remove-10.1, VerbNet encodes not only NP V NP PP.Source
PP.Destination (\emph{Doug removed the smudges from the tabletop}) but also NP
V NP (\emph{Doug removed the smudges}). \verbenet{} only includes the first
one; it is understood that the second one can be automatically inferred from
the first one, without being (manually) validated\footnote{However, this
principle concerning sub-frames is not applied for verbs which accept a single
double-locative complement ``from here to there (a single complement PP.source
PP.destination)''  without accepting a single source complement (PP.source),
while accepting a single destination complement (PP.destination) : \emph{Fred a
transferré le vin de la cruche en pierre vers la cruche en terre cuite  (Fred
transferred the wine from the stone jar to the terra-cotta jar), *Fred a
transferré le vin de la cruche en pierre (*Fred transferred the wine from the
stone jar), Fred a transferré le vin vers la cruche en terre cuite (Fred
transferred the wine to the terra-cotta  jar.}}.

The second one concerns the order of the complements. VerbNet sometimes encodes
two frames which differ only by the order of the complements, e.g. in bring
11-3 the frames NP V NP PP.destination (\emph{Nora brought the book to the
meeting}) and NP V PP.destination NP (\emph{Nora brought to lunch the book}).
In French, the order of complements  depends on a number of syntactic and
semantic factors \citep{thuilier2012contraintes}, but it doesn't seem that
it depends on a lexical factor, i.e. what is the lexical verb governing the
complements. As a consequence, \verbenet{} only records one frame in such
cases, e.g. it only records \emph{NP V NP PP.destination} (\emph{Nora a
apporté le livre au meeting}) with the direct object before the PP; it is
understood that the other frame, \emph{NP V  PP.destination NP} (\emph{Nora a
apporté au meeting le livre}) can be automatically inferred from the first
one.

\subsubsection{Case by case work}\label{subsubsec:casebycase}

In some cases, the second step in building \verbenet{} is hard. There are two
main reasons for that. First, there exist semantic differences which are taken
into account in VerbNet but not in LVF or in LG. For example, among the verbs
of Sending and Carrying (VerbNet super-class 11), the verbs in  classes 11.3,
11.4 and 11.5 describe an accompanied motion (both the Agent and the Theme
change location as in \emph{Pamela drove packages to NY}), while those in
classes 11.1 and 11.2 describe an unaccompanied motion (only the Theme changes
location as in \emph{Pamela sent packages to NY}). In the French resources,
classes do exist for verbs with a change of location for a Theme caused by an
Agent, but nothing is said about the Agent being or not being in motion.  In
the face  of this difficulty, two solutions are possible: either make a study
of French verbs of sending and carrying to distinguish accompanied and
unaccompanied motions, or simply ignore this semantic difference. We opted for
the second solution since this semantic difference does not appear to be
relevant for a task such as semantic role labeling.\footnote{Moreover, it seems
that, for some English verbs, the Agent can be moving or not as reflected by
the difference between VerbNet's classes 11.4 and 11.4-1.} Ignoring this
semantic difference leads us to adopt in \verbenet{} a hierarchy for verbs of
Sending and Carrying different from that in VerbNet: there is no equivalent in
\verbenet{} of  class 11.4, the verbs belonging to this class being added to
11.1 or 11.2. Let us add that there is no French equivalent of class 11.3 made
up of the two verbs \emph{bring} and \emph{take} with a deictically-specified
direction \citep[page 135]{levin1993english} since the French locative
deictic \emph{ici} and \emph{là} don't work as \emph{here} and
\emph{there}\footnote{\emph{Je suis là} (lit. \emph{I am there}) can mean
\emph{Je suis ici} (lit \emph{I am here}).}.

The second main source of difficulty comes from crucial differences between
French and English. There exist translation problems between these two
languages which are very well-known and documented, for example translation of
motion verbs as illustrated in  \emph{John swam across the river} $\rightarrow$
\emph{Jean a traversé la rivière à la nage} (lit. John crossed the river with
the swim). We put aside those well-known cases here to discuss  more subtle
difficulties as illustrated with the verbs of change of possession. In VerbNet,
there exist ten classes dedicated to these verbs. It seems that such a
hierarchy cannot be adopted for French. Without going into all the details, let
us underline the following points:

\begin{itemize}

    \item  The absence of dative and benefactive alternations in French means that
    the difference between VerbNet's classes 13.1 and 13.2 should probably not be
    kept.

    \item  The semantic difference between 13.1 and 13.3 (namely HAS-POSSESSION
    versus FUTURE-POSSESSION) is perhaps too subtle and could be ignored.

    \item  The preposition \emph{with} in the frame corresponding to \emph{Agent V
    Recipient {with} Theme} used in 13.4-1 and 13.4-2 has to be replaced with
    \emph{en} and/or \emph{de} according to the verb (e.g. \emph{Luc livre Max
    en/*de lait}, \emph{Luc équipe Max en/de téléviseurs}, \emph{Luc dote Max
    *en/de téléviseurs}), which requires a reorganization into (sub-)classes.

\end{itemize}

All in all, it turns out that entering into the frame details has led us to
revise the hierarchy of \verbenet{} though we are trying to minimize the amount
of revision in order to keep as much semantic information from VerbNet as
possible.

\subsection{Troisième étape}
\label{third}

La troisième étape, qui n'a pas encore commencé, est de valider manuellement
pour chaque classes les verbes proposés par correspondance de ressources en
supprimant les verbes erronés et en rajoutant les verbes manquant à fin que la
ressource ait été entièrement validée manuellement.

\section{Conclusion} We have presented a method for adapting the English
syntactic and semantic resource VerbNet to a new language. This method combines
the automation of structures transfer, automatic translation of the lexicon and
linguistic expertise. We have applied this method to French and have reached a
state where it is validated and the systematic work on each class is currently
being realized.  We are not  able to give an evaluation of this resource since
it is not yet completed. When it will be completed, we will make it freely
available along with the web-based tool which makes it possible to
collaboratively edit it.

In this work, we acknowledge the structural differences existing between
languages: the class structure of \verbenet{} does not follow exactly VerbNet.
We keep track of such changes so that the differences between the two resources
are explicit and well-documented. Thus they will be available for interacting
with other resources through mappings, making our resource useful for
multilingual applications.

This work is part of the
ASFALDA\footnote{https://sites.google.com/site/anrasfalda/} project which goals
include the creation of a French FrameNet and mappings between it and other
semantic resources, like LVF, LG and \verbenet{}.
%Some goals of the ASFALDA project we are members of are to produce
%FrameNet-based semantic role labeling tools and a mapping between LVF and
%FrameNet. Even is ASFALDA emphasis is on supervised SRL using a French
%annotated corpus, our own knowledge-based approach based on \verbenet{} will
%be complementary thanks to the mappings we are drawing between \verbenet{} and
%LVF.
